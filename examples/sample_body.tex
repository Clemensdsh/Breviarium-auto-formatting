% ==========================================
% 自动生成自 output.csv
% 不要手动编辑此文件,请修改 CSV 后重新生成
% ==========================================

\psPrintToc
\clearpage
\pagenumbering{arabic}
\pagestyle{fancy}
\begin{paracol}{2}
\psHeaderOneCap{In Nativitate Domini}{耶穌聖誕日課}
\psHeaderOneLowercase{Ad I Vesperas}{申正經 (係前一日)}
\psRubric{(Dícitur Pater noster et Ave María, secréto.)}{默念:天主經、聖母經。\\ (畢,明聲念:)}
\psVR{V}{Deus, in adiutórium meum inténde.}{天主惟專於我扶祐。}
\psVR{R}{Dómine, ad adjuvándum me festína.}{主速格以救助我。}
\psGloria{Glória Patri, et Fílio, et Spirítui Sancto. Sicut erat in princípio, et nunc, et semper, et in sǽcula sæculórum. Amen. Allelúja.}{欽頌榮福於罷德肋、及費畧、及斯彼利多三多,\\若今茲、若永遠、及無窮世。亞孟。亞勒路亞。}
\psAntiphonRepeat{Rex pacificus magnificatus est, cujus vultum desiderat universa terra.}{和睦王已丕顯,遍地顒慕覲厥容。}
\psPsalmTitle{Psalmus 109}{聖詠109}
\psVerse{Dixit Dóminus Dómino meo: * \\Sede a dextris meis:}{主語吾主曰:\\爾坐予右,}
\psVerse{Donec ponam inimícos tuos, * \\scabéllum pedum tuórum.}{俟予伏爾仇\\於爾足下。}
\psVerse{Virgam virtútis tuæ emíttet Dóminus ex Sion: * \\domináre in médio inimicórum tuórum.}{主自西婉將發爾德能之鋌,\\爾將王爾仇之中。}
\psVerse{Tecum princípium in die virtútis tuæ in splendóribus sanctórum: * \\ex útero ante lucíferum génui te.}{爾德能之日,原始偕爾諸聖之光中;\\啟明之先,予已胚胎生爾。}
\psVerse{Jurávit Dóminus, et non pœnitébit eum: * \\Tu es sacérdos in ætérnum secúndum órdinem Melchísedech.}{主矢而不悔:\\爾將永享撒責爾鐸德爵位,從默其塞德例。}
\psVerse{Dóminus a dextris tuis, * \\confrégit in die iræ suæ reges.}{主於爾右,\\厥怒之日,已勦衆師。}
\psVerse{Judicábit in natiónibus, implébit ruínas: * \\conquassábit cápita in terra multórum.}{將審判衆民,補諸殘缺,\\而於普地摧諸渠魁。}
\psVerse{De torrénte in via bibet: * \\proptérea exaltábit caput.}{途中將飲洶流之水,\\乃得昂首。}
\psGloria{Glória Patri, et Fílio, * et Spirítui Sancto. \\Sicut erat in princípio, et nunc, et semper, * et in sǽcula sæculórum. Amen.}{欽頌榮福於罷德肋、及費畧、及斯彼利多三多,\\若今茲、若永遠、及無窮世。亞孟。}
\psAntiphonRepeat{Rex pacificus magnificatus est, cujus vultum desiderat universa terra.}{和睦王已丕顯,遍地顒慕覲厥容。}
\psAntiphonRepeat{Magnificátus est Rex Pacíficus super omnes reges terræ.}{和睦王已丕顯,越普世衆王。}
\psPsalmTitle{Psalmus 110}{聖詠110}
\psVerse{Confitébor tibi, Dómine, in toto corde meo: * \\in consílio justórum, et congregatióne.}{主,予以全心將讚美主,\\於義人會議之處。}
\psVerse{Magna ópera Dómini: * \\exquisíta in omnes voluntátes ejus.}{主工弘矣,\\盡合厥志。}
\psVerse{Conféssio et magnificéntia opus ejus: * \\et justítia ejus manet in sǽculum sǽculi.}{備全稱頌即厥工也,博施亦厥工也,\\而義永存。}
\psVerse{Memóriam fecit mirabílium suórum, miséricors et miserátor Dóminus: * \\escam dedit timéntibus se.}{仁慈主已憶其昔行靈異,\\賜食於畏主者。}
\psVerse{Memor erit in sǽculum testaménti sui: * \\virtútem óperum suórum annuntiábit pópulo suo:}{主永不忘厥詔,\\將示厥工之德能於其民,}
\psVerse{Ut det illis hereditátem géntium: * \\ópera mánuum ejus véritas, et judícium.}{以與伊等異教之嗣業。\\主之掌握,惟直惟判;}
\psVerse{Fidélia ómnia mandáta ejus: confirmáta in sǽculum sǽculi, * \\facta in veritáte et æquitáte.}{教令信實,永遠世騐;\\証皆真皆公。}
\psVerse{Redemptiónem misit pópulo suo: * \\mandávit in ætérnum testaméntum suum.}{遣救贖於衆民,\\頒詔於無窮世。}
\psVerse{Sanctum, et terríbile nomen ejus: * \\inítium sapiéntiæ timor Dómini.}{厥名聖哉赫哉。\\欽畏即明哲之原,}
\psVerse{Intelléctus bonus ómnibus faciéntibus eum: * \\laudátio ejus manet in sǽculum sǽculi.}{循其明哲而行,乃大益。\\伊等稱頌,尊於永世。}
\psGloria{Glória Patri, et Fílio, * et Spirítui Sancto. \\Sicut erat in princípio, et nunc, et semper, * et in sǽcula sæculórum. Amen.}{欽頌榮福於罷德肋、及費畧、及斯彼利多三多,\\若今茲、若永遠、及無窮世。亞孟。}
\psAntiphonRepeat{Magnificátus est Rex Pacíficus super omnes reges terræ.}{和睦王已丕顯,越普世衆王。}
\psAntiphonRepeat{Compléti sunt dies Maríæ, ut páreret fílium suum primogénitum.}{瑪利亞產厥首子,期已滿。}
\psPsalmTitle{Psalmus 111}{聖詠111}
\psVerse{Beátus vir, qui timet Dóminum: * \\in mandátis ejus volet nimis.}{畏天主真福人哉,\\歡忭而奉詔令。}
\psVerse{Potens in terra erit semen ejus: * \\generátio rectórum benedicétur.}{嗣將強於世,\\義人後代將蒙殊福。}
\psVerse{Glória, et divítiæ in domo ejus: * \\et justítia ejus manet in sǽculum sǽculi.}{居世且榮且富,\\而厥義永存。}
\psVerse{Exórtum est in ténebris lumen rectis: * \\miséricors, et miserátor, et justus.}{質直者居暗已受光照,\\且惻且義。}
\psVerse{Jucúndus homo qui miserétur et cómmodat, dispónet sermónes suos in judício: * \\quia in ætérnum non commovébitur.}{天主哀矜,濟以財,普樂斯人哉。審判將慎備厥語,\\盖永不易動。}
\psVerse{In memória ætérna erit justus: * \\ab auditióne mala non timébit.}{義人永憶於無窮世,\\將不懼惡聲。}
\psVerse{Parátum cor ejus speráre in Dómino, confirmátum est cor ejus: * \\non commovébitur donec despíciat inimícos suos.}{厥心預備既定,以望天主,\\不易動,目懾厥仇。}
\psVerse{Dispérsit, dedit paupéribus: justítia ejus manet in sǽculum sǽculi, * \\cornu ejus exaltábitur in glória.}{伊散施於貧者,義且永世,\\毅德上徹於榮光。}
\psVerse{Peccátor vidébit, et irascétur, déntibus suis fremet et tabéscet: * \\desidérium peccatórum períbit.}{罪人觀即瞋怒,且切齒而狺憤,鬱而摧裂;\\罪人之冀望將亡。}
\psGloria{Glória Patri, et Fílio, * et Spirítui Sancto. \\Sicut erat in princípio, et nunc, et semper, * et in sǽcula sæculórum. Amen.}{欽頌榮福於罷德肋、及費畧、及斯彼利多三多,\\若今茲、若永遠、及無窮世。亞孟。}
\psAntiphonRepeat{Compléti sunt dies Maríæ, ut páreret fílium suum primogénitum.}{瑪利亞產厥首子,期已滿。}
\psAntiphonRepeat{Scitóte quia prope est regnum Dei: amen dico vobis, quia non tardábit.}{宜知天國邇,實語爾等將匪遙。}
\psPsalmTitle{Psalmus 112}{聖詠112}
\psVerse{Laudáte, púeri, Dóminum: * \\laudáte nomen Dómini.}{幼童讚美主,\\讚美主名。}
\psVerse{Sit nomen Dómini benedíctum, * \\ex hoc nunc, et usque in sǽculum.}{主名宜稱揚,\\若今至於永世。}
\psVerse{A solis ortu usque ad occásum, * \\laudábile nomen Dómini.}{自日出至日入處,\\主名宜稱揚。}
\psVerse{Excélsus super omnes gentes Dóminus, * \\et super cælos glória ejus.}{主萬衆之上,\\巍哉厥榮;諸天之上哉。}
\psVerse{Quis sicut Dóminus Deus noster, qui in altis hábitat, * \\et humília réspicit in cælo et in terra?}{有如吾主天主,平居上,\\而睠顧在天在地下之微者。}
\psVerse{Súscitans a terra ínopem, * \\et de stércore érigens páuperem:}{自下地復拯空乏者,\\而自污處攸起困抑者;}
\psVerse{Ut cóllocet eum cum princípibus, * \\cum princípibus pópuli sui.}{置之與公侯,\\偕厥民之公侯。}
\psVerse{Qui habitáre facit stérilem in domo, * \\matrem filiórum lætántem.}{使胎荒者居室內,\\而為樂多子之母。}
\psGloria{Glória Patri, et Fílio, * et Spirítui Sancto. \\Sicut erat in princípio, et nunc, et semper, * et in sǽcula sæculórum. Amen.}{欽頌榮福於罷德肋、及費畧、及斯彼利多三多,\\若今茲、若永遠、及無窮世。亞孟。}
\psAntiphonRepeat{Scitóte quia prope est regnum Dei: amen dico vobis, quia non tardábit.}{宜知天國邇,實語爾等將匪遙。}
\psAntiphonRepeat{Leváte cápita vestra: ecce appropinquábit redémptio vestra.}{舉目翹首,爾等真福即邇。}
\psPsalmTitle{Psalmus 116}{聖詠116}
\psVerse{Laudáte Dóminum, omnes gentes: * \\laudáte eum, omnes pópuli:}{請諸異教者稱頌主,\\衆民稱頌主。}
\psVerse{Quóniam confirmáta est super nos misericórdia ejus: * \\et véritas Dómini manet in ætérnum.}{盖主慈堅定於我等,\\而純誠存於無窮世。}
\psGloria{Glória Patri, et Fílio, * et Spirítui Sancto. \\Sicut erat in princípio, et nunc, et semper, * et in sǽcula sæculórum. Amen.}{欽頌榮福於罷德肋、及費畧、及斯彼利多三多,\\若今茲、若永遠、及無窮世。亞孟。}
\psAntiphonRepeat{Leváte cápita vestra: ecce appropinquábit redémptio vestra.}{舉目翹首,爾等真福即邇。}
\psRubric{Capitulum}{節目}
\psHeaderThree{Tit. 3:4-5}{鐸三4,5}
\psText{Appáruit grátia Dei Salvatóris nostri ómnibus homínibus, erúdiens nos, ut abnegántes impietátem, et sæculária desidéria, sóbrie, et juste, et pie vivámus in hoc sǽculo.}{吾救世天主為人者,仁慈攸顯矣,非為吾儕立義功而來,乃因厥慈救我等。}
\psRubric{Hymnus}{聖歌}
\psHymnHeader{Jesu, Redémptor ómnium}{救世耶穌}
\psHymnStanza{Jesu, Redémptor ómnium, Quem lucis ante oríginem, \\Parem patérnæ glóriæ, Pater suprémus édidit.}{救世耶穌,啟明之先,\\上父攸生,與已均榮。}
\psHymnStanza{Tu lumen, et splendor Patris, Tu spes perénnis ómnium: \\Inténde quas fundunt preces Tui per orbem sérvuli.}{爾乃父光,蒼生之望,\\普世微僕,攸祈俯聽。}
\psHymnStanza{Meménto, rerum Cónditor, Nostri quod olim córporis, \\Sacráta ab alvo Vírginis Nascéndo, formam súmpseris.}{肇基萬有,請記昔者,\\由童女胎,降取吾身。}
\psHymnStanza{Testátur hoc præsens dies, Currens per anni círculum, \\Quod solus a sinu Patris Mundi salus advéneris.}{週歲届期,今日作証,\\爾自父懷,降來救世。}
\psHymnStanza{Hunc astra, tellus, æquora, Hunc omne, quod cælo subest, \\Salútis auctórem novæ Novo salútant cántico.}{星地海幽,寰宇萬有,\\新咏讚主,復新人元。}
\psHymnStanza{Et nos, beáti, quos sacra Rigávit unda sánguinis, \\Natális ob diem tui, Hymni tribútum sólvimus.}{吾儕獲贖,因爾寶血,\\念爾聖誕,和賡稱頌。}
\psHymnStanza{Jesu, tibi sit glória, Qui natus es de Vírgine, \\Cum Patre et almo Spíritu, In sempitérna sǽcula. Amen.}{誕於童女,耶穌永福,\\歸爾偕父,聖神世世。亞孟。}
\psVR{V}{Crástina die delébitur iníquitas terræ.}{旦日世孽將泯。}
\psVR{R}{Et regnábit super nos Salvátor mundi.}{救世者將至吾儕。}
\psRubric{Ad Magnificat}{聖母歌}
\psAntiphonRepeat{Cum ortus fúerit sol de cælo, vidébitis Regem regum procedéntem a Patre, tamquam sponsum de thálamo suo.}{太陽已出,爾輩將見由父諸王之王,如新出於閤。}
\psVerse{Magníficat ánima mea Dóminum: \\et exsultávit spíritus meus in Deo salutári meo.}{感頌吾主吾神,\\無任忻愉於救我者。}
\psVerse{Quia respéxit humilitátem ancíllæ suæ: \\ecce enim ex hoc beátam me dicent omnes generatiónes.}{緣其垂顧婢子之微,\\後人亦將於我乎讚頌矣。}
\psVerse{Quia fecit mihi magna qui potens est: \\et sanctum nomen ejus.}{夫全能者大展厥德於我,\\錫以異恩用彰聖名。}
\psVerse{Et misericórdia ejus a progénie in progénies: \\timéntibus eum.}{仁慈無量,將沿世世,\\於諸畏敬之者。}
\psVerse{Fecit poténtiam in bráchio suo: \\dispérsit supérbos mente cordis sui.}{以厥臂神力顯大能,\\麾彼驕盈。}
\psVerse{Depósuit poténtes de sede: \\et exaltávit húmiles.}{黜彼尊者於高位,\\而陟舉夫謙遜者。}
\psVerse{Esuriéntes implévit bonis: \\et dívites dimísit inánes.}{饑虛以福實之,\\飫滿以傾棄之。}
\psVerse{Suscépit Israel púerum suum: \\recordátus misericórdiæ suæ.}{且不忘大慈,\\賜以其子,}
\psVerse{Sicut locútus est ad patres nostros: \\Ábraham, et sémini ejus in sǽcula.}{以踐所許於吾祖\\亞罷郎,及後世之子孫者。}
\psGloria{Glória Patri, et Fílio, * et Spirítui Sancto. \\Sicut erat in princípio, et nunc, et semper, * et in sǽcula sæculórum. Amen.}{欽頌榮福於罷德肋、及費畧、及斯彼利多三多,\\若今茲、若永遠、及無窮世。亞孟。}
\psAntiphonRepeat{Cum ortus fúerit sol de cælo, vidébitis Regem regum procedéntem a Patre, tamquam sponsum de thálamo suo.}{太陽已出,爾輩將見由父諸王之王,如新出於閤。}
\psRubric{Oratio}{祝文}
\psCollect{Concéde, quǽsumus, omnípotens Deus: ut nos Unigéniti tui nova per carnem Natívitas líberet; quos sub peccáti jugo vetústa sérvitus tenet. Per eúndem Dóminum nostrum Jesum Christum Filium tuum: Qui tecum vivit et regnat in unitate Spiritus Sancti Deus, per omnia saecula saeculorum. Amen.}{懇祈全能天主,彼久在罪軛,賴爾惟一子肉身新誕,幸救脫亦。為聖子耶穌基利斯督我等主,偕聖父及聖神,為一天主,永生永王。亞孟。}
\psVR{V}{Dóminus vobíscum.}{主與爾偕焉。}
\psVR{R}{Et cum spíritu tuo.}{並於爾神。}
\psVR{V}{Benedicámus Dómino.}{讚美主。}
\psVR{R}{Deo grátias.}{謝天主。}
\psVR{V}{Fidélium ánimæ per misericórdiam Dei requiéscant in pace.}{凡諸信者靈魂,賴天主仁慈息止安所。}
\psVR{R}{Amen.}{亞孟。}
\psRubric{(Pater noster, secreto.)}{(後默念天主經)}
\psThickRule
\psHeaderOneLowercase{Ad Matutinum}{夜課經}
\psRubric{(Dícitur Pater noster, Ave María, et Credo, secréto.)}{默念:天主經、聖母經、信經。\\ (畢,明聲念:)}
\psVR{V}{Dómine, lábia mea apéries.}{主啟我唇。}
\psVR{R}{Et os meum annuntiábit laudem tuam.}{我音將頌揚主。}
\psVR{V}{Deus, in adiutórium meum inténde.}{天主惟專於我扶祐。}
\psVR{R}{Dómine, ad adjuvándum me festína.}{主速格以救助我。}
\psGloria{Glória Patri, et Fílio, et Spirítui Sancto. Sicut erat in princípio, et nunc, et semper, et in sǽcula sæculórum. Amen. Allelúja.}{欽頌榮福於罷德肋、及費畧、及斯彼利多三多。\\若今茲、若永遠、及無窮世。亞孟。亞勒路亞。}
\psRubric{Invitatorium}{請經}
\psAntiphonRepeat{Christus natus est nobis: * Veníte, adorémus.}{基利斯督降誕為吾,請欽崇。}
\psPsalmTitle{Psalmus 94}{聖詠94 (請向主踴躍歌呼)}
\psVerse{Veníte, exsultémus Dómino, \\jubilémus Deo, salutári nostro: \\præoccupémus fáciem ejus in confessióne, \\et in psalmis jubilémus ei.}{請向主踴躍歌呼,\\欣籲救世天主聖容前,\\預備痛告,\\和唱聖詠。}
\psRubric{Repetitur}{重念}
\psText{Christus natus est nobis: * Veníte, adorémus.}{欽崇基利斯督為吾降誕,請欽崇。}
\psVerse{Quóniam Deus magnus Dóminus, \\et Rex magnus super omnes deos, \\quóniam non repéllet Dóminus plebem suam: \\quia in manu ejus sunt omnes fines terræ, \\et altitúdines móntium ipse cónspicit.}{主為惟一大主,\\王一切諸神,\\且不棄厥民。\\全地毬皆握厥手,\\而俯視崔嵬於下。}
\psRubric{Repetitur}{重念}
\psText{Veníte, adorémus.}{基利斯督為吾降誕,請欽崇。}
\psVerse{Quóniam ipsíus est mare, et ipse fecit illud, \\et áridam fundavérunt manus ejus: \\veníte, adorémus, et procidámus ante Deum: \\plorémus coram Dómino, qui fecit nos, \\quia ipse est Dóminus, Deus noster; \\nos autem pópulus ejus, et oves páscuæ ejus.}{盖奄有大海,即其攸造者;\\且制創大地矣。\\真吾主,請欽崇跪俯伏天主臺前,\\涕泣造我者我等天主,\\我等為其民,\\及其屬牧之羊。}
\psRubric{Repetitur}{重念}
\psText{Christus natus est nobis: * Veníte, adorémus.}{欽崇基利斯督為吾降誕,請欽崇。}
\psVerse{Hódie, si vocem ejus audiéritis, \\nolíte obduráre corda vestra, \\sicut in exacerbatióne secúndum diem tentatiónis in desérto: \\ubi tentavérunt me patres vestri, \\probavérunt et vidérunt ópera mea.}{今日幸聽主言,\\勿愎爾心,\\如昔日忤命疑悔於曠野。\\彼處爾祖疑試,\\且目擊其事。}
\psRubric{Repetitur}{重念}
\psText{Veníte, adorémus.}{基利斯督為吾降誕,請欽崇。}
\psVerse{Quadragínta annis próximus fui generatióni huic, et dixi: \\Semper hi errant corde; \\ipsi vero non cognovérunt vias meas: \\quibus jurávi in ira mea: \\Si introíbunt in réquiem meam.}{予邇此輩四十載,\\予乃曰:彼卒世錯誤,\\不知吾諸道。\\予瞋怒矢志:\\伊等萬不得進予安旅。}
\psRubric{Repetitur}{重念}
\psText{Christus natus est nobis: * Veníte, adorémus.}{欽崇基利斯督為吾降誕,請欽崇。}
\psGloria{Glória Patri, et Fílio, * et Spirítui Sancto. \\Sicut erat in princípio, et nunc, et semper, * et in sǽcula sæculórum. Amen.}{欽頌榮福於罷德肋、及費畧、及斯彼利多三多,\\若今茲、若永遠、及無窮世。亞孟。}
\psRubric{Repetitur}{重念}
\psText{Veníte, adorémus.}{基利斯督為吾降誕,請欽崇。}
\psRubric{Repetitur}{重念}
\psText{Christus natus est nobis: * Veníte, adorémus.}{基利斯督降誕為吾,請欽崇。}
\psRubric{Hymnus}{聖歌}
\psHymnStanza{Jesu, Redémptor ómnium, Quem lucis ante oríginem, \\Parem patérnæ glóriæ, Pater suprémus édidit.}{救世耶穌,啟明之先,\\上父攸生,與已均榮。}
\psHymnStanza{Tu lumen, et splendor Patris, Tu spes perénnis ómnium: \\Inténde quas fundunt preces Tui per orbem sérvuli.}{爾乃父光,蒼生之望,\\普世微僕,攸祈俯聽。}
\psHymnStanza{Meménto, rerum Cónditor, Nostri quod olim córporis, \\Sacráta ab alvo Vírginis Nascéndo, formam súmpseris.}{肇基萬有,請記昔者,\\由童女胎,降取吾身。}
\psHymnStanza{Testátur hoc præsens dies, Currens per anni círculum, \\Quod solus a sinu Patris Mundi salus advéneris.}{週歲届期,今日作証,\\爾自父懷,降來救世。}
\psHymnStanza{Hunc astra, tellus, æquora, Hunc omne, quod cælo subest, \\Salútis auctórem novæ Novo salútant cántico.}{星地海幽,寰宇萬有,\\新咏讚主,復新人元。}
\psHymnStanza{Et nos, beáti, quos sacra Rigávit unda sánguinis, \\Natális ob diem tui, Hymni tribútum sólvimus.}{吾儕獲贖,因爾寶血,\\念爾聖誕,和賡稱頌。}
\psHymnStanza{Jesu, tibi sit glória, Qui natus es de Vírgine, \\Cum Patre et almo Spíritu, In sempitérna sǽcula. Amen.}{誕於童女,耶穌永福,\\歸爾偕父,聖神世世。亞孟。}
\psHeaderTwo{In I Nocturno}{夜第一節}
\psAntiphonRepeat{Dóminus dixit ad me: Fílius meus es tu, ego hódie génui te.}{主語予:爾乃吾子,予今日生爾。}
\psPsalmTitle{Psalmus 2}{聖詠2}
\psVerse{Quare fremuérunt gentes: * \\et pópuli meditáti sunt inánia?}{異教者因何嗔叱,\\而多衆徒謀在虛?}
\psVerse{Astitérunt reges terræ, et príncipes convenérunt in unum * \\advérsus Dóminum, et advérsus Christum ejus.}{世王鉅公合一謀,\\主並厥基利斯督。}
\psVerse{Dirumpámus víncula eórum: * \\et projiciámus a nobis jugum ipsórum.}{吾儕且破厥桎梏,\\且棄於我董厥軛。}
\psVerse{Qui hábitat in cælis, irridébit eos: * \\et Dóminus subsannábit eos.}{居天者將誚而哂,\\伊等維時震怒,}
\psVerse{Tunc loquétur ad eos in ira sua, * \\et in furóre suo conturbábit eos.}{諭之而忿然,\\侮辱之。}
\psVerse{Ego autem constitútus sum Rex ab eo super Sion montem sanctum ejus, * \\prǽdicans præcéptum ejus.}{予於伊奉遣,為西婉聖山之王,\\傳厥教令。}
\psVerse{Dóminus dixit ad me: * \\Fílius meus es tu, ego hódie génui te.}{主諭予曰:\\爾為予子,今日生爾。}
\psVerse{Póstula a me, et dabo tibi gentes hereditátem tuam, * \\et possessiónem tuam términos terræ.}{祈予,予將與爾異教者爾嗣業,\\爾權至於遍地之末境。}
\psVerse{Reges eos in virga férrea, * \\et tamquam vas fíguli confrínges eos.}{以鐵枝治伊等,\\而如窑瓷將裂。}
\psVerse{Et nunc, reges, intellígite: * \\erudímini, qui judicátis terram.}{今諸王宜聽,\\判世者且知之。}
\psVerse{Servíte Dómino in timóre: * \\et exsultáte ei cum tremóre.}{宜事主而欽畏,\\宜向主而躍且顫。}
\psVerse{Apprehéndite disciplínam, nequándo irascátur Dóminus, * \\et pereátis de via justa.}{操守規責,恐主怒,\\聽爾離於義道。}
\psVerse{Cum exárserit in brevi ira ejus: * \\beáti omnes qui confídunt in eo.}{厥忿一時驟熾,\\凡望主皆福人哉。}
\psGloria{Glória Patri, et Fílio, * et Spirítui Sancto. \\Sicut erat in princípio, et nunc, et semper, * et in sǽcula sæculórum. Amen.}{欽頌榮福於罷德肋、及費畧、及斯彼利多三多,\\若今茲、若永遠、及無窮世。亞孟。}
\psAntiphonRepeat{Dóminus dixit ad me: Fílius meus es tu, ego hódie génui te.}{主語予:爾乃吾子,予今日生爾。}
\psAntiphonRepeat{Tamquam sponsus Dóminus procédens de thálamo suo.}{主如新出於閤。}
\psPsalmTitle{Psalmus 18}{聖詠18}
\psVerse{Cæli enárrant glóriam Dei: * \\et ópera mánuum ejus annúntiat firmaméntum.}{諸天頌揚天主榮,\\堅定天傳示厥掌之工。}
\psVerse{Dies diéi erúctat verbum, * \\et nox nocti índicat sciéntiam.}{日遞傳於日,\\而夜遞於夜,指其知。}
\psVerse{Non sunt loquélæ, neque sermónes, * \\quorum non audiántur voces eórum.}{無語無詞,\\不聞其音。}
\psVerse{In omnem terram exívit sonus eórum: * \\et in fines orbis terræ verba eórum.}{其聲已傳遍地,\\而厥訓言達於坤輿之四陲。}
\psVerse{In sole pósuit tabernáculum suum: * \\et ipse tamquam sponsus procédens de thálamo suo:}{主置其幕庭於太陽,\\而袨麗如新出於閤,}
\psVerse{Exsultávit ut gigas ad curréndum viam, * \\a summo cælo egréssio ejus:}{迅如巨人馳騁。\\出自至上天,}
\psVerse{Et occúrsus ejus usque ad summum ejus: * \\nec est qui se abscóndat a calóre ejus.}{而會於其至峻,\\無有避其熱。}
\psVerse{Lex Dómini immaculáta, convértens ánimas: * \\testimónium Dómini fidéle, sapiéntiam præstans párvulis.}{主道無污,化變靈性;\\主証忠實,使幼童智。}
\psVerse{Justítiæ Dómini rectæ, lætificántes corda: * \\præcéptum Dómini lúcidum, illúminans óculos.}{主判真實,悅樂諸心;\\主令光明,照牖衆目。}
\psVerse{Timor Dómini sanctus, pérmanens in sǽculum sǽculi: * \\judícia Dómini vera, justificáta in semetípsa.}{畏主之德,聖德也,永世恒存;\\主判真實而自義。}
\psVerse{Desiderabília super aurum et lápidem pretiósum multum: * \\et dulcióra super mel et favum.}{當愛慕踰於金寶,\\而甘於蜜。}
\psVerse{Étenim servus tuus custódit ea, * \\in custodiéndis illis retribútio multa.}{盖爾僕守之守之,\\其報甚厚。}
\psVerse{Delícta quis intélligit? ab occúltis meis munda me: * \\et ab aliénis parce servo tuo.}{誰自悟過愆?望滌於予隱暗,\\而以他人之故赦免爾僕。}
\psVerse{Si mei non fúerint domináti, tunc immaculátus ero: * \\et emundábor a delícto máximo.}{若罪不為我主,予將無污,\\而滌潔其大諐。}
\psVerse{Et erunt ut compláceant elóquia oris mei: * \\et meditátio cordis mei in conspéctu tuo semper.}{予言語將合旨,\\而心默存時,在主臺前。}
\psVerse{Dómine, adjútor meus, * \\et redémptor meus.}{主予扶祐,\\及予救贖。}
\psGloria{Glória Patri, et Fílio, * et Spirítui Sancto. \\Sicut erat in princípio, et nunc, et semper, * et in sǽcula sæculórum. Amen.}{欽頌榮福於罷德肋、及費畧、及斯彼利多三多,\\若今茲、若永遠、及無窮世。亞孟。}
\psAntiphonRepeat{Tamquam sponsus Dóminus procédens de thálamo suo.}{主如新出於閤。}
\psAntiphonRepeat{Diffúsa est grátia in lábiis tuis: proptérea benedíxit te Deus in ætérnum.}{爾唇備美,故天主降福於爾世世。}
\psPsalmTitle{Psalmus 44}{聖詠44}
\psVerse{Eructávit cor meum verbum bonum: * \\dico ego ópera mea Regi.}{予心已噯嘉言,\\以予攸製語王。}
\psVerse{Lingua mea cálamus scribæ: * \\velóciter scribéntis.}{予舌\\如揮毫,}
\psVerse{Speciósus forma præ fíliis hóminum, diffúsa est grátia in lábiis tuis: * \\proptérea benedíxit te Deus in ætérnum.}{生民以來,惟爾純美。諸美備於爾唇,\\故天主降福永世。}
\psVerse{Accíngere gládio tuo super femur tuum, * \\potentíssime.}{大能者\\請繫爾刀,}
\psVerse{Spécie tua et pulchritúdine tua: * \\inténde, próspere procéde, et regna.}{因爾令儀令容。\\祈盼視降福,行且王。}
\psVerse{Propter veritátem, et mansuetúdinem, et justítiam: * \\et dedúcet te mirabíliter déxtera tua.}{因真實及善良及義者,\\爾右手引爾於奇異。}
\psVerse{Sagíttæ tuæ acútæ, pópuli sub te cadent: * \\in corda inimicórum regis.}{爾箭鋒利,\\入於王仇之心,萬民手下僵仆。}
\psVerse{Sedes tua, Deus, in sǽculum sǽculi: * \\virga directiónis virga regni tui.}{天主寶座無窮世,\\厥國柄權真且直。}
\psVerse{Dilexísti justítiam, et odísti iniquitátem: * \\proptérea unxit te Deus, Deus tuus, óleo lætítiæ præ consórtibus tuis.}{爾已嗜義而嫉惡,\\故天主爾天主,在爾侣中以福樂聖油擦爾。}
\psVerse{Myrrha, et gutta, et cásia a vestiméntis tuis, a dómibus ebúrneis: * \\ex quibus delectavérunt te fíliæ regum in honóre tuo.}{彌辣及烏答及加細亞,由其袍服,並由象牙宮殿。\\諸王多女悅慕欽崇爾,}
\psVerse{Ástitit regína a dextris tuis in vestítu deauráto: * \\circúmdata varietáte.}{母后五彩金衣\\侍於右。}
\psVerse{Audi fília, et vide, et inclína aurem tuam: * \\et oblivíscere pópulum tuum et domum patris tui.}{吾女宜視宜聽,且傾其耳,\\而忘乃民與爾父之家。}
\psVerse{Et concupíscet Rex decórem tuum: * \\quóniam ipse est Dóminus Deus tuus, et adorábunt eum.}{王即愛厥儀容,\\厥是爾主天主,伊等將拜伏之。}
\psVerse{Et fíliæ Tyri in munéribus * \\vultum tuum deprecabúntur: omnes dívites plebis.}{第樂多女子獻義,\\萬民諸富者欲欽崇。}
\psVerse{Omnis glória ejus fíliæ Regis ab intus, * \\in fímbriis áureis circumamícta varietátibus.}{王女子之榮華在於內,\\金縈其衣,周圍五彩。}
\psVerse{Adducéntur Regi vírgines post eam: * \\próximæ ejus afferéntur tibi.}{隨行衆童女將引進之,\\其次者亦引至之。}
\psVerse{Afferéntur in lætítia et exsultatióne: * \\adducéntur in templum Regis.}{見引至王踴躍,\\引至王臺前。}
\psVerse{Pro pátribus tuis nati sunt tibi fílii: * \\constítues eos príncipes super omnem terram.}{代厥祖已生多子,\\將置彼衆為遍地鉅公。}
\psVerse{Mémores erunt nóminis tui: * \\in omni generatióne et generatiónem.}{憶念爾名\\無窮世,}
\psVerse{Proptérea pópuli confitebúntur tibi in ætérnum: * \\et in sǽculum sǽculi.}{多衆讚美\\於無窮世。}
\psGloria{Glória Patri, et Fílio, * et Spirítui Sancto. \\Sicut erat in princípio, et nunc, et semper, * et in sǽcula sæculórum. Amen.}{欽頌榮福於罷德肋、及費畧、及斯彼利多三多,\\若今茲、若永遠、及無窮世。亞孟。}
\psAntiphonRepeat{Diffúsa est grátia in lábiis tuis: proptérea benedíxit te Deus in ætérnum.}{爾唇備美,故天主降福於爾世世。}
\psVR{V}{Tamquam sponsus Dóminus procédens de thálamo suo.}{主如新出於閤。}
\psVR{R}{Exsultávit ut gigas ad curréndum viam.}{迅如巨人馳騁。}
\psRubric{Pater noster (totum dicitur secreto)}{默念天主經 (在天云云)}
\psVR{V}{Et ne nos indúcas in tentatiónem.}{而不我許陷於誘感。}
\psVR{R}{Sed líbera nos a malo.}{乃救我於凶惡。}
\psRubric{Absolutio}{畢經}
\psCollect{Exáudi, Dómine Jesu Christe, preces servórum tuórum, et miserére nobis: \\qui cum Patre et Spíritu Sancto vivis et regnas in sǽcula sæculórum.}{吾主耶穌基利斯督,俯聽吾僕輩祈禱,而矜憐我等,其偕罷德肋、偕費畧、偕斯彼利多三多,乃生乃王無窮世。亞孟。}
\psVR{R}{Amen.}{亞孟。}
\psRubric{Benedictio}{降福}
\psVR{V}{Jube, domne, benedícere.}{請降福。}
\psCollect{Benedictióne perpétua benedícat nos Pater ætérnus.}{永父以永福降恩澤於我等。}
\psVR{R}{Amen.}{亞孟。}
\psLesson{Lectio I: Isaias 9:1-6}{書一 (依撒義亞前知者九章)}
\psRubric{}{此書不念其題目}
\psText{Primo témpore alleviáta est terra Zábulon, et terra Néphtali: et novíssimo aggraváta est via maris trans Jordánem Galilǽæ géntium. Pópulus, qui ambulábat in ténebris, vidit lucem magnam: habitántibus in regióne umbræ mortis, lux orta est eis. Multiplicásti gentem, et non magnificásti lætítiam. Lætabúntur coram te, sicut qui lætántur in messe: sicut exsúltant victóres capta præda, quando dívidunt spólia. Jugum enim óneris ejus, et virgam húmeri ejus, et sceptrum exactóris ejus superásti sicut in die Mádian. Quia omnis violénta prædátio cum tumúltu, et vestiméntum mistum sánguine, erit in combustiónem, et cibus ignis. Párvulus enim natus est nobis, et fílius datus est nobis, et factus est principátus super húmerum ejus: et vocábitur nomen ejus Admirábilis, Consiliárius, Deus, Fortis, Pater futúri sǽculi, Princeps pacis.}{初時雜布鸞及搦大利地,肩負已輕,而適近過若而當海道。異教者加理勒亞即困重矣。彼衆昔行暗中,已覩巨光;寓死影之地而顯光。人雖加多,然弗加多樂。伊等至爾前將樂,如樂大有,如踴躍分克敵之獲。盖爾已脫厥重軛,厥肩荷並司稅者權,如瑪弟昂日。盖凡克獲閧然,與血衣將焚為火食。嬰孩為吾儕誕矣,而子已付於吾儕矣。渠王在厥肩,厥名稱奇異,恭贊天主,勇後世之父,和平之首領。}
\psVR{R}{Hódie nobis cælórum Rex de Vírgine nasci dignátus est, ut hóminem pérditum ad cæléstia regna revocáret: * \\Gaudet exércitus Angelórum: quia salus ætérna humáno géneri appáruit.}{今日天主不棄,誕於童女,以救淪喪,復歸天國。\\天朝師旅諸神欣樂,盖永福已丕顯。}
\psRubric{Benedictio}{降福}
\psText{Unigénitus Dei Fílius nos benedícere et adjuváre dignétur. Amen.}{望天主惟一子降福及扶持我等。}
\psLesson{Lectio II: Isaias 40:1-8}{書二}
\psText{Consolámini, consolámini, pópule meus, dicit Deus vester. Loquímini ad cor Jerúsalem, et advocáte eam: quóniam compléta est malítia ejus, remíssa est iníquitas ejus: suscépit de manu Dómini duplícia pro ómnibus peccátis suis. Vox clamántis in desérto: Paráte viam Dómini, rectas fácite in solitúdine sémitas Dei nostri. Omnis vallis exaltábitur, et omnis mons et collis humiliábitur, et erunt prava in dirécta, et áspera in vias danas. Et revelábitur glória Dómini: et vidébit omnis caro páriter quod os Dómini locútum est. Vox dicéntis: Clama. Et dixi: Quid clamábo? Omnis caro fœnum, et omnis glória ejus quasi flos agri. Exsiccátum est fœnum, et cécidit flos, quia spíritus Dómini sufflávit in eo. Vere fœnum est pópulus: exsiccátum est fœnum, et cécidit flos: verbum autem Dómini nostri manet in ætérnum.}{爾等天主曰:予民互慰互慰,語曰路撒稜於心,而還迎伊。盖厥惡既盡,厥孽已赦,為衆孽獲倍於主手。聲呼曠野,開治主道,正直主徑。充填空谷,蕩夷山陵,曲衺既直,崎嶇既坦。主榮將露,凡負形者明見主口攸語。厥聲曰:呼號。予曰:呼號何?人皆草芥,厥榮若華,草枯華落,盖主神嘘彼焉。斯民誠為草莽,草芥萎而華謝,天主話永存。}
\psVR{R}{Hódie nobis de cælo pax vera descéndit: * \\Hódie per totum mundum mellíflui facti sunt cæli.}{今日真和平為吾自天降,\\今日普世諸天降蜜。}
\psRubric{Benedictio}{降福}
\psText{Spíritus Sancti grátia illúminet sensus et corda nostra. Amen.}{降福:聖神之寵照衆人司及心。}
\psLesson{Lectio III: Isaias 52:1-6}{書三}
\psText{Consúrge, consúrge, indúere fortitúdinem tuam, Sion, indúere vestiménta glóriæ tuæ, Jerúsalem, cívitas sancti: quia non adjíciet ultra ut pertránseat per te incircúmcisus, et immúndus. Excutere de púlvere, consúrge, sede, Jerúsalem: solve víncula colli tui, captíva fília Sion. Quia hæc dicit Dóminus: Gratis venúmdati estis, et sine argénto redimémini. Quia hæc dicit Dóminus Deus: In Ægýptum descéndit pópulus meus in princípio, ut colónus esset ibi: et Assur absque ulla causa calumniátus est eum. Et nunc quid mihi est hic, dicit Dóminus, quóniam abláta est plebs mea gratis? Dominatóres ejus iníque agunt, dicit Dóminus, et júgiter tota die nomen meum blasphemátur. Propterea sciet pópulus meus nomen meum in die illa: quia ego ipse qui loquébar, ecce adsum.}{西婉興起興起,衣爾勇;主城曰路撒稜,衣爾榮之衣。夫不割損及污者,不經由爾。曰路撒稜振塵起坐,西婉被擄女子,解爾頸索。主曰:汝輩昔徒鬻,今無貲蒙贖。又曰:初予民下厄日多居為氓,亞斯爾無故肆侮。主曰:予奚有於斯?盖予民被徒奪,王伊者稔惡,終刺辱予名。故予民彼日將知,昔予語茲,親在焉。}
\psVR{R}{Quem vidístis, pastóres? dícite, annuntiáte nobis: quis appáruit in terris? * \\Natum vídimus, et choros Angelórum collaudántes Dóminum.}{牧童奚見?述告我儕。\\見已生者,天神合讚主。}
\psRubric{Absolutio}{畢經}
\psCollect{Ipsiús píetas et misericórdia nos ádjuvet, \\qui cum Patre et Spíritu Sancto vivit et regnat in sǽcula sæculórum.}{主仁慈扶祐我等,\\偕罷德肋、偕斯彼利多三多,乃生乃王永世。}
\psVR{R}{Amen.}{亞孟。}
\psRubric{Benedictio}{降福}
\psVR{V}{Jube, domne, benedícere.}{請降福。}
\psCollect{Deus Pater omnípotens sit nobis propítius et clemens.}{乞全能天主父盼視寬容我等。}
\psVR{R}{Amen.}{亞孟。}
\psThickRule
\psHeaderTwo{In II Nocturno}{夜第二節}
\psAntiphonRepeat{Suscépimus, Deus, misericórdiam tuam in médio templi tui.}{天主吾衆已接主慈於爾殿中。}
\psPsalmTitle{Psalmus 47}{聖詠47}
\psVerse{Magnus Dóminus, et laudábilis nimis * \\in civitáte Dei nostri, in monte sancto ejus.}{主溥哉,至當頌\\於吾天主城內,於厥聖山。}
\psVerse{Fundátur exsultatióne univérsæ terræ mons Sion, * \\látera Aquilónis, cívitas Regis magni.}{置基西婉山,遍地忻愉,\\北及左右皆大王之城。}
\psVerse{Deus in dómibus ejus cognoscétur, * \\cum suscípiet eam.}{天主收錄堅城時,\\即認識於城中之諸殿宇。}
\psVerse{Quóniam ecce reges terræ congregáti sunt: * \\convenérunt in unum.}{伊等主會齊觀視,\\且共聚一處。}
\psVerse{Ipsi vidéntes sic admiráti sunt, conturbáti sunt, commóti sunt: * \\tremor apprehéndit eos.}{異駭不寧,\\遍地驚怖。}
\psVerse{Ibi dolóres ut parturiéntis: * \\in spíritu veheménti cónteres naves Tharsis.}{痗痃痛悸,如娩乳然。\\使風大作,將破達斯衆舟。}
\psVerse{Sicut audívimus, sic vídimus in civitáte Dómini virtútum, in civitáte Dei nostri: * \\Deus fundávit eam in ætérnum.}{昔攸聞斯目擊,於德能天主城,於吾天主城。\\天主創基彼城於無窮世。}
\psVerse{Suscépimus, Deus, misericórdiam tuam, * \\in médio templi tui.}{主吾衆已接王慈\\於主殿中。}
\psVerse{Secúndum nomen tuum, Deus, sic et laus tua in fines terræ: * \\justítia plena est déxtera tua.}{天主如主名,即實頌揚主於遍地諸境,\\主掌滿布公義。}
\psVerse{Lætétur mons Sion, et exsúltent fíliæ Judæ: * \\propter judícia tua, Dómine.}{主因厥攸判,西婉山宜慶慰,\\而孺達諸女子躍然。}
\psVerse{Circúmdate Sion, et complectímini eam: * \\narráte in túrribus ejus.}{請周圍西婉,而四方環繞\\於厥樓。}
\psVerse{Pónite corda vestra in virtúte ejus: * \\et distribúite domos ejus, ut enarrétis in progénie áltera.}{稱述殫思締於厥德能,人每分列厥樓,\\使可述於後世。}
\psVerse{Quóniam hic est Deus, Deus noster in ætérnum et in sǽculum sǽculi: * \\ipse reget nos in sǽcula.}{盖斯即是天主,吾天主於無窮世,\\斯即御我等無窮世。}
\psGloria{Glória Patri, et Fílio, * et Spirítui Sancto. \\Sicut erat in princípio, et nunc, et semper, * et in sǽcula sæculórum. Amen.}{欽頌榮福於罷德肋、及費畧、及斯彼利多三多,\\若今茲、若永遠、及無窮世。亞孟。}
\psAntiphonRepeat{Suscépimus, Deus, misericórdiam tuam in médio templi tui.}{天主吾衆已接爾仁慈於爾殿中。}
\psAntiphonRepeat{Oriétur in diébus ejus justítia, et abundántia pacis.}{主之日將享大太平而王。}
\psPsalmTitle{Psalmus 71}{聖詠71}
\psVerse{Deus, judícium tuum Regi da: * \\et justítiam tuam Fílio Regis:}{望天主以判斷之權與王,\\而以主義德之用與王子。}
\psVerse{Judicáre pópulum tuum in justítia, * \\et páuperes tuos in judício.}{可使以義審判厥衆,\\而以公明判爾諸煢獨者。}
\psVerse{Suscípiant montes pacem pópulo: * \\et colles justítiam.}{喬岳望接和平為衆民,\\而培塿均接義德。}
\psVerse{Judicábit páuperes pópuli, et salvos fáciet fílios páuperum: * \\et humiliábit calumniatórem.}{將判斷庶民,衆貧拯救,貧困衆子,\\且抑誣妄者。}
\psVerse{Et permanébit cum sole, et ante lunam, * \\in generatióne et generatiónem.}{偕日將而月先\\存於無窮世。}
\psVerse{Descéndet sicut plúvia in vellus: * \\et sicut stillicídia stillántia super terram.}{降如雨於羢毛,\\又如霡霂入土。}
\psVerse{Oriétur in diébus ejus justítia, et abundántia pacis: * \\donec auferátur luna.}{維時義德將出,及普世和平,\\至於月亡。}
\psVerse{Et dominábitur a mari usque ad mare: * \\et a flúmine usque ad términos orbis terrárum.}{自海至海將王,\\自江河至寰宇之境界。}
\psVerse{Coram illo prócident Æthíopes: * \\et inimíci ejus terram lingent.}{厥前厄第阿彼亞人跪伏地,\\而厥仇餂地。}
\psVerse{Reges Tharsis, et ínsulæ múnera ófferent: * \\reges Árabum et Saba dona addúcent.}{達臘斯衆王及島嶼奉獻,\\亞辣彼亞及撒把王來獻。}
\psVerse{Et adorábunt eum omnes reges terræ: * \\omnes gentes sérvient ei.}{遍地諸王將致敬,\\異教者咸役事。}
\psVerse{Quia liberábit páuperem a poténte: * \\et páuperem, cui non erat adjútor.}{盖將救貧人於強者之手,\\即無扶持者之貧人。}
\psVerse{Parcet páuperi et ínopi: * \\et ánimas páuperum salvas fáciet.}{赦免貧困,\\而拯救衆貧人靈;}
\psVerse{Ex usúris et iniquitáte rédimet ánimas eórum: * \\et honorábile nomen eórum coram illo.}{救贖厥靈於貸利者及於不義者,\\其名厥前丕顯。}
\psVerse{Et vivet, et dábitur ei de auro Arábiæ, et adorábunt de ipso semper: * \\tota die benedícent ei.}{又將活矣,將與之亞拉被亞金,欽崇之,\\終日讚美之。}
\psVerse{Et erit firmaméntum in terra in summis móntium, superextollétur super Líbanum fructus ejus: * \\et florébunt de civitáte sicut fœnum terræ.}{將為穩定於地山巔上,其實卓越利巴諾,\\而於城堡發花蕃茂如菅茅然。}
\psVerse{Sit nomen ejus benedíctum in sǽcula: * \\ante solem pérmanet nomen ejus.}{厥名應讚美於無窮世,\\厥名已存太陽先。}
\psVerse{Et benedicéntur in ipso omnes tribus terræ: * \\omnes gentes magnificábunt eum.}{因而蒼生被福,\\異教者舉揚伊。}
\psVerse{Benedíctus Dóminus, Deus Israël, * \\qui facit mirabília solus:}{讚美依撒厄爾天主,\\獨一能行靈異者。}
\psVerse{Et benedíctum nomen majestátis ejus in ætérnum: * \\et replébitur majestáte ejus omnis terra: fiat, fiat.}{並永遠讚美其威嚴之名,厥威嚴而遍地將滿。\\是是。}
\psGloria{Glória Patri, et Fílio, * et Spirítui Sancto. \\Sicut erat in princípio, et nunc, et semper, * et in sǽcula sæculórum. Amen.}{欽頌榮福於罷德肋、及費畧、及斯彼利多三多,\\若今茲、若永遠、及無窮世。亞孟。}
\psAntiphonRepeat{Oriétur in diébus ejus justítia, et abundántia pacis.}{主之日將享大太平而王。}
\psAntiphonRepeat{Véritas de terra orta est, et justítia de cælo prospéxit.}{真者由地出,義者自天俯顧。}
\psPsalmTitle{Psalmus 84}{聖詠84}
\psVerse{Benedixísti, Dómine, terram tuam: * \\avertísti captivitátem Jacob.}{主已降福爾地,\\救贖雅歌伯,}
\psVerse{Remisísti iniquitátem plebis tuæ: * \\operuísti ómnia peccáta eórum.}{赦免諸民惡行,\\遮蔽其諸罪愆。}
\psVerse{Mitigásti omnem iram tuam: * \\avertísti ab ira indignatiónis tuæ.}{平厥怒\\而霽其愠顏。}
\psVerse{Convérte nos, Deus, salutáris noster: * \\et avérte iram tuam a nobis.}{望救吾天主,賜我回轉,\\而平厥怒。}
\psVerse{Numquid in ætérnum irascéris nobis? * \\aut exténdes iram tuam a generatióne in generatiónem?}{豈將永遠怒我乎?\\或延爾怒世世乎?}
\psVerse{Deus, tu convérsus vivificábis nos: * \\et plebs tua lætábitur in te.}{天主盼視將活我等,\\爾民將歡慰於爾。}
\psVerse{Osténde nobis, Dómine, misericórdiam tuam: * \\et salutáre tuum da nobis.}{望主示我等爾慈,\\而與我等爾救世者。}
\psVerse{Audiam quid loquátur in me Dóminus Deus: * \\quóniam loquétur pacem in plebem suam.}{傾聽吾主語,何事於予?\\盖將語和平於庶民,}
\psVerse{Et super sanctos suos: * \\et in eos, qui convertúntur ad cor.}{及於諸聖,\\與實心歸正然。}
\psVerse{Verúmtamen prope timéntes eum salutáre ipsíus: * \\ut inhábitet glória in terra nostra.}{救世者邇於敬畏之者,\\使光榮存於吾地。}
\psVerse{Misericórdia, et véritas obviavérunt sibi: * \\justítia, et pax osculátæ sunt.}{仁慈也真實也相遇,\\義也和也相親。}
\psVerse{Véritas de terra orta est: * \\et justítia de cælo prospéxit.}{真者由地出,\\義者自天俯顧。}
\psVerse{Étenim Dóminus dabit benignitátem: * \\et terra nostra dabit fructum suum.}{主將賜仁慈,\\而吾地將發厥實。}
\psVerse{Justítia ante eum ambulábit: * \\et ponet in via gressus suos.}{義也將發於前,\\而足遂踐行之。}
\psGloria{Glória Patri, et Fílio, * et Spirítui Sancto. \\Sicut erat in princípio, et nunc, et semper, * et in sǽcula sæculórum. Amen.}{欽頌榮福於罷德肋、及費畧、及斯彼利多三多,\\若今茲、若永遠、及無窮世。亞孟。}
\psAntiphonRepeat{Véritas de terra orta est, et justítia de cælo prospéxit.}{真者由地出,義者自天俯顧。}
\psVR{V}{Speciósus forma præ fíliis hóminum.}{人子中邁種。}
\psVR{R}{Diffúsa est grátia in lábiis tuis.}{爾唇備美。}
\psRubric{Pater noster (totum dicitur secreto)}{默念天主經 (在天云云)}
\psVR{V}{Et ne nos indúcas in tentatiónem.}{而不我許陷於誘感。}
\psVR{R}{Sed líbera nos a malo.}{乃救我於凶惡。}
\psRubric{Absolutio}{畢經}
\psCollect{Ipsiús píetas et misericórdia nos ádjuvet, \\qui cum Patre et Spíritu Sancto vivit et regnat in sǽcula sæculórum.}{主仁慈扶祐我等,\\偕罷德肋、偕斯彼利多三多,乃生乃王永世。}
\psVR{R}{Amen.}{亞孟。}
\psRubric{Benedictio}{降福}
\psVR{V}{Jube, domne, benedícere.}{請降福。}
\psCollect{Deus Pater omnípotens sit nobis propítius et clemens.}{乞全能天主父盼視寬容我等。}
\psVR{R}{Amen.}{亞孟。}
\psLesson{Lectio IV: Sermo sancti Leónis Papæ}{書四 (聖良教皇講論)}
\psText{Salvátor noster, dilectíssimi, hódie natus est: gaudeámus. Neque enim fas est locum esse tristítiæ, ubi natális est vitæ; quam consúmpto mortalitátis timóre, ingéreret nobis de promíssa æternitáte lætítiam. Nemo ab hujus alacritátis participatióne secérnitur. Una cunctis lætítiæ commúnis est rátio: quia Dóminus noster peccáti mortísque destrúctor, sicut nullum a culpa líberum répperit, ita venit liberatúrus omnes. Exsúltet sanctus, quia propínquat ad palmam: gáudeat peccátor, quia invitátur ad véniam: animétur gentílis, quia vocátur ad vitam. Dei enim Fílius secúndum plenitúdinem témporis, quam divíni consílii inscrutábilis altitúdo dispósuit, reconciliándam auctóri suo natúram géneris assúmpsit humáni, ut invéntor mortis diábolus, per ipsam quam vícerat, vincerétur.}{仁昆,今日吾救世降誕也,請偕樂。夫憂虞於生命之誕,不相並焉。死懼已亡,吾輩乃望將進攸許常生之樂域矣。於斯同樂,無有或岐,緣樂義皆同。而吾主已怯死滅愆,誰為無罪?故降來拯衆也。聖人宜樂,將蒙召報;罪人宜樂,將蒙宥赦;異教宜發志,將蒙召得常生。天主子預宣奧旨,所定救世期已至。欲人類合於天主,取人性而為人,使邪魔以所得勝即以之敗亡。}
\psVR{R}{O magnum mystérium, et admirábile sacraméntum, ut animália vidérent Dóminum natum, jacéntem in præsépio: * \\Beáta Virgo, cujus víscera meruérunt portáre Dóminum Christum.}{異哉奧義,畜生見主誕於櫪。 * \\福哉童女之腹,幸懷主基利斯督。}
\psVR{V}{Ave, María, grátia plena: Dóminus tecum.}{附經:亞物瑪利亞滿被額辣濟亞者,主與爾偕焉。}
\psRubric{Repetitur}{重念}
\psText{Beáta Virgo, cujus víscera meruérunt portáre Dóminum Christum.}{福哉童女之腹,幸懷主基利斯督。}
\psRubric{Benedictio}{降福}
\psVR{V}{Jube, domne, benedícere.}{請降福。}
\psCollect{Christus perpétuæ det nobis gáudia vitæ.}{望基利斯督賜吾常生之樂。}
\psVR{R}{Amen.}{亞孟。}
\psLesson{Lectio V}{書五}
\psText{Igitur cum omnipotens Deus ad occurrendum hosti nequissimo, non in virtute majestatis, sed in infirmitate carnis nostrae prælium iniret; eam naturam ei congreditur, eamque formam, quae et in divinitate non minueretur, et in humanitate non consumeretur. Quia enim in solo Domino nostro, qui sine peccato natus est, nulla fuit contagio reatus humani; non tamen in natura sancta, sicut in carne peccati, lex est posita peccati. Ad hanc ergo conceptionem, ad hanc partum, nulla carnalis concupiscentia, nulla lex peccati, nulla lex mortis intervenit. Virgo regia ex stirpe David eligitur, quae sacra graviditate gestanda, divinam humanamque prolem prius conciperet mente, quam corpore. Et ne superni ignara consilii, ad inusitatos paveret affatus, quod in ea operandum erat a Spiritu Sancto, colloquiis discit angelicis: nec damnum credit pudoris, Dei Genetrix mox futura.}{於斯奧戎,為吾輩大彰公平之義。全能天主征伐虐仇,韜厥性,巍乃用吾性之微賤。雖取吾屬壞之性與人同,然不染罪孽。所謂雖一日生世之孩,未有無污,豈可與此聖誕同論哉?聖誕絕無肉身僻情與罪污原染矣。為天主而人之母,乃選童女,達未得聖王苗裔,孕懷先於心後於形也。童女罔諳主義,聞天神報而訝之。天主遣神直告,匪由人道,乃以聖神之能,童貞依然毫不受損。}
\psVR{R}{Beáta Dei Génitrix María, cujus víscera intácta permansérunt: * \\Hódie génuit Salvatórem sǽculi.}{福哉天主母,瑪利亞童貞如故。 * \\今日生救世者。}
\psVR{V}{Beáta, quæ credidísti, quóniam perfécta sunt ómnia, quæ dicta sunt tibi a Dómino.}{附經:因信福哉,盖主語於伊成畢。}
\psRubric{Repetitur}{重念}
\psText{Hódie génuit Salvatórem sǽculi.}{今日生救世者。}
\psRubric{Benedictio}{降福}
\psVR{V}{Jube, domne, benedícere.}{請降福。}
\psCollect{Ignem sui amóris accéndat Deus in córdibus nostris.}{祈天主燃於吾本愛之火。}
\psVR{R}{Amen.}{亞孟。}
\psLesson{Lectio VI}{書六}
\psText{Agámus ergo, dilectíssimi, grátias Deo Patri, per Fílium ejus, in Spíritu Sancto, qui propter nimiam caritátem suam, qua diléxit nos, misertus est nostri: et cum essémus mórtui peccátis, convivificávit nos Christo, ut essémus in ipso nova creatúra, novúmque figméntum. Deponámus ergo véterem hóminem cum áctibus suis: et adépti participatiónem generatiónis Christi, carnis renuntiémus opéribus. Agnósce, o Christiáne, dignitátem tuam: et divínæ consors factus natúræ, noli in véterem vilitátem degéneri conversatióne redíre. Meménto cujus cápitis, et cujus córporis sis membrum. Reminiscere quia erútus de potestáte tenebrárum, translátus es in Dei lumen, et regnum.}{仁昆,宜讚謝天主父子聖神,愛我等至深,矜憐吾向已死於罪,復活於基利斯督,使得成新造之美工。今宜易舊人與其僻情,盖既與耶穌共其生,應絕肉情矣。奉教者宜識爾高爵,既通主性,勿行不義,歸復於原微陋鄙賤之事。亦憶為何首何體之肢,深記脫出魔手,徙天國無疆之永光。}
\psVR{R}{Sancta et immaculáta virgínitas, quibus te láudibus éfferam, néscio: * \\Quia quem cæli cápere non póterant, tuo grémio contulísti.}{維聖維童貞,應何如廖頌舉揚爾予罔知。 * \\盖諸天不克容者,以爾胎抱懷。}
\psVR{V}{Benedícta tu in muliéribus, et benedíctus fructus ventris tui.}{附經:女中爾為讚美,並爾胎實為讚美。}
\psRubric{Repetitur}{重念}
\psText{Quia quem cæli cápere non póterant, tuo grémio contulísti.}{盖諸天不克容者,以爾胎抱懷。}
\psGloria{Glória Patri, et Fílio, et Spirítui Sancto.}{欽頌榮福於罷德肋、及費畧、及斯彼利多三多。}
\psRubric{Repetitur}{重念}
\psText{Quia quem cæli cápere non póterant, tuo grémio contulísti.}{盖諸天不克容者,以爾胎抱懷。}
\psThickRule
\psHeaderTwo{In III Nocturno}{夜第三節}
\psAntiphonRepeat{Ipse invocábit me: Allelúja, Pater meus es tu, Allelúja.}{伊將呼號予亞勒路亞,爾乃予父亞勒路亞。}
\psPsalmTitle{Psalmus 88}{聖詠88}
\psVerse{Misericórdias Dómini * \\in ætérnum cantábo.}{予永稱頌主多仁慈,\\予口世世顯揚爾真實。}
\psVerse{In generatiónem et generatiónem * \\annuntiábo veritátem tuam in ore meo.}{盖爾已語仁慈將成於天,\\爾真實建定於彼。}
\psVerse{Quóniam dixísti: In ætérnum misericórdia ædificábitur in cælis: * \\præparábitur véritas tua in eis.}{予排定嗣業於預錄者,已誓予僕達未德,\\將備爾嗣於無窮世,永世加爾座。}
\psVerse{Dispósui testaméntum eléctis meis, jurávi David servo meo: * \\Usque in ætérnum præparábo semen tuum.}{主諸天顯揚爾諸靈異,\\爾真實於衆聖之會。}
\psVerse{Et ædificábo in generatiónem et generatiónem * \\sedem tuam.}{在天疇伴天主?\\天主子中疇得彷彿乎?}
\psVerse{Confitebúntur cæli mirabília tua, Dómine: * \\étenim veritátem tuam in ecclésia sanctórum.}{天主光榮於衆聖之會,\\弘矣嚴矣,超諸厥周之上。}
\psVerse{Quóniam quis in núbibus æquábitur Dómino: * \\símilis erit Deo in fíliis Dei?}{德能天主疇比爾?\\爾乃能真實周爾。}
\psVerse{Deus, qui glorificátur in consílio sanctórum: * \\magnus et terríbilis super omnes qui in circúitu ejus sunt.}{爾王海,能命息厥浪。\\爾壓傲者如被創,因爾德能之臂散亡諸仇。}
\psVerse{Dómine, Deus virtútum, quis símilis tibi? * \\potens es, Dómine, et véritas tua in circúitu tuo.}{諸天及地乃爾物,爾置基坤輿及其實。\\造北方暨海,達波爾及厄爾滿山因爾名踴躍。}
\psVerse{Tu domináris potestáti maris: * \\motum autem flúctuum ejus tu mítigas.}{爾德能之臂,\\望爾手穩定,右手舉揚。}
\psVerse{Tu humiliásti sicut vulnerátum, supérbum: * \\in bráchio virtútis tuæ dispersísti inimícos tuos.}{義也判也爾座之基,\\仁及真先導爾。}
\psVerse{Tui sunt cæli, et tua est terra, orbem terræ et plenitúdinem ejus tu fundásti: * \\aquilónem, et mare tu creásti.}{前知忻愉者真福人哉。\\主伊等因爾容之光將行,}
\psVerse{Thabor et Hermon in nómine tuo exsultábunt: * \\tuum bráchium cum poténtia.}{因爾名終日踴躍,而因爾義舉揚。\\盖主乃厥德之榮,因主旨吾勇峻舉。}
\psVerse{Firmétur manus tua, et exaltétur déxtera tua: * \\justítia et judícium præparátio sedis tuæ.}{吾扶祐於主,\\於聖依撒厄爾吾王。}
\psVerse{Misericórdia et véritas præcédent fáciem tuam: * \\beátus pópulus, qui scit jubilatiónem.}{維時主明啓諸聖曰:予扶祐於能者,\\於民中已舉預錄者。}
\psVerse{Dómine, in lúmine vultus tui ambulábunt, et in nómine tuo exsultábunt tota die: * \\et in justítia tua exaltabúntur.}{已獲予僕達未德,\\以聖油傅伊。}
\psVerse{Quóniam glória virtútis eórum tu es: * \\et in beneplácito tuo exaltábitur cornu nostrum.}{予手將扶予膊,堅伊仇讐,\\毫無得勝者,惡逆子毋得肆害。}
\psVerse{Quia Dómini est assúmptio nostra: * \\et Sancti Israël, regis nostri.}{伊前予將戮厥仇,\\凡惡伊者使奔伊。}
\psVerse{Tunc locútus es in visióne sanctis tuis, et dixísti: * \\Pósui adjutórium in poténte: et exaltávi eléctum de plebe mea.}{予實予慈偕伊,\\因予名厥勇高舉。}
\psVerse{Invéni David, servum meum: * \\óleo sancto meo unxi eum.}{予將置厥手於海,\\厥右手於江河。}
\psVerse{Manus enim mea auxiliábitur ei: * \\et bráchium meum confortábit eum.}{伊將呼予:\\爾乃子父,予天主及吾保。}
\psVerse{Nihil profíciet inimícus in eo: * \\et fílius iniquitátis non appónet nocére ei.}{予即俾伊為元子,\\昂然王世者。}
\psVerse{Et concídam a fácie ipsíus inimícos ejus: * \\et odiéntes eum in fugam convértam.}{永存予慈於伊,\\予詔一定。}
\psVerse{Et véritas mea, et misericórdia mea cum ipso: * \\et in nómine meo exaltábitur cornu ejus.}{令厥代世世,\\厥座如天日。}
\psVerse{Et ponam in mari manum ejus: * \\et in flúminibus déxteram ejus.}{若厥子方予令弗踐予道,\\若輕予諸義者而不守予令,}
\psVerse{Ipse invocábit me: Pater meus es tu: * \\Deus meus, et suscéptor salútis meæ.}{用枝責厥不義,\\笞撻厥罪。}
\psVerse{Et ego primogénitum ponam illum: * \\excélsum præ régibus terræ.}{但不盡泯予慈,\\又弗傷伊。}
\psVerse{In ætérnum servábo illi misericórdiam meam: * \\et testaméntum meum fidéle ipsi.}{因予真實,又不忽予詔,\\凡出予口固不虛。}
\psVerse{Et ponam in sǽculum sǽculi semen ejus: * \\et thronum ejus sicut dies cæli.}{予因予聖曾一誓非誣,達未德厥嗣永存,\\厥座予前如日如月,望世世忠証在天。}
\psVerse{Si autem derelíquerint fílii ejus legem meam: * \\et in judíciis meis non ambuláverint.}{爾退逖而忽爾基利斯督,滅厥詔,\\污聖堂於地,毁厥諸籬,其安居乃為惧。}
\psVerse{Si justítias meas profanáverint: * \\et mandáta mea non custodíerint:}{凡行路者掠檢,\\為厥隣者譏侮。}
\psVerse{Visitábo in virga iniquitátes eórum: * \\et in verbéribus peccáta eórum.}{舉壓伊者右手,\\令厥仇歡樂,弗扶厥刀,戰時弗祐。}
\psVerse{Misericórdiam autem meam non dispérgam ab eo: * \\neque nocébo in veritáte mea:}{止厥蠲沉於地,\\厥座短折,時日侮辱伊。}
\psVerse{Neque profanábo testaméntum meum: * \\et quæ procédunt de lábiis meis, non fáciam írrita.}{主至終時震怒乎?\\爾怒如火熾乎?}
\psVerse{Semel jurávi in sancto meo, si David méntiar: * \\semen ejus in ætérnum manébit.}{憶予體何如?\\造人衆徒然乎?}
\psVerse{Et thronus ejus sicut sol in conspéctu meo, * \\et sicut luna perfécta in ætérnum: et testis in cælo fidélis.}{誰人生而不死,\\出脫厥靈於獄手乎?}
\psVerse{Tu vero repulísti et despexísti: * \\distulísti Christum tuum.}{主昔時爾慈何在?\\如於達未德真誓。}
\psVerse{Evertísti testaméntum servi tui: * \\profanásti in terra sanctuárium ejus.}{記爾諸僕被辱於異教多衆,\\予存於予懷。}
\psVerse{Destruxísti omnes sepes ejus: * \\posuísti firmaméntum ejus formídinem.}{主爾仇覿面言,\\覿面言改變爾基利斯督。}
\psVerse{Diripuérunt eum omnes transeúntes viam: * \\factus est oppróbrium vicínis suis.}{讚美主於無窮。\\是是。}
\psVerse{Exaltásti déxteram depriméntium eum: * \\lætificásti omnes inimícos ejus.}{欽頌榮福於罷德肋、及費畧、及斯彼利多三多,\\若今茲、若永遠、及無窮世。亞孟。}
\psAntiphonRepeat{Ipse invocábit me: Allelúja, Pater meus es tu, Allelúja.}{伊將呼號予亞勒路亞,爾乃予父亞勒路亞。}
\psAntiphonRepeat{Læténtur cæli, et exsúltet terra ante fáciem Dómini, quóniam venit.}{上天應歡慶,下地鼓舞主容前,盖已臨。}
\psPsalmTitle{Psalmus 95}{聖詠95}
\psVerse{Cantáte Dómino cánticum novum: * \\cantáte Dómino, omnis terra.}{請衆向主同奏新歌,\\遍地請奏。}
\psVerse{Cantáte Dómino, et benedícite nómini ejus: * \\annuntiáte de die in diem salutáre ejus.}{向主奏而稱頌主名,\\日每傳示厥救世者。}
\psVerse{Annuntiáte inter gentes glóriam ejus, * \\in ómnibus pópulis mirabília ejus.}{傳示異教者厥榮,\\於衆民厥異行。}
\psVerse{Quóniam magnus Dóminus, et laudábilis nimis: * \\terríbilis est super omnes deos.}{盖主大矣,宜頌揚;\\至矣,赫赫於諸僞尊之上。}
\psVerse{Quóniam omnes dii géntium dæmónia: * \\Dóminus autem cælos fecit.}{異教偽主皆邪魔,\\惟主即造諸天之主。}
\psVerse{Conféssio, et pulchritúdo in conspéctu ejus: * \\sanctimónia et magnificéntia in sanctificatióne ejus.}{主臺前拜颺,穆穆然皇皇然,\\恪恭溥博,即為欽崇聖堂。}
\psVerse{Afférte Dómino, pátriæ géntium, afférte Dómino glóriam et honórem: * \\afférte Dómino glóriam nómini ejus.}{請萬民獻主,獻主榮福且尊敬,\\向其名獻尊敬。}
\psVerse{Tóllite hóstias, et introíte in átria ejus: * \\adoráte Dóminum in átrio sancto ejus.}{齎上祭物而進厥堂,\\欽崇主於厥聖庭。}
\psVerse{Commoveátur a fácie ejus univérsa terra: * \\dícite in géntibus quia Dóminus regnávit.}{主臺前遍地宜震動。\\爾等向異教者曰:主王矣,}
\psVerse{Étenim corréxit orbem terræ qui non commovébitur: * \\judicábit pópulos in æquitáte.}{盖已化定天下,將永不變易,\\其以義判萬民。}
\psVerse{Læténtur cæli, et exsúltet terra, commoveátur mare, et plenitúdo ejus: * \\gaudébunt campi, et ómnia quæ in eis sunt.}{上天應歡慶,下地鼓舞,海宇且感動與其所包涵者。\\凡阡陌之中皆熙熙然,}
\psVerse{Tunc exsultábunt ómnia ligna silvárum a fácie Dómini, quia venit: * \\quóniam venit judicáre terram.}{彼時山林諸木主前翔舞。\\盖已降臨,降臨審判遍地。}
\psVerse{Judicábit orbem terræ in æquitáte, * \\et pópulos in veritáte sua.}{判天下以義,\\而衆人以真。}
\psGloria{Glória Patri, et Fílio, * et Spirítui Sancto. \\Sicut erat in princípio, et nunc, et semper, * et in sǽcula sæculórum. Amen.}{欽頌榮福於罷德肋、及費畧、及斯彼利多三多,\\若今茲、若永遠、及無窮世。亞孟。}
\psAntiphonRepeat{Læténtur cæli, et exsúltet terra ante fáciem Dómini, quóniam venit.}{上天應歡慶,下地鼓舞主容前,盖已臨。}
\psAntiphonRepeat{Notum fecit Dóminus, Allelúja, salutáre suum, Allelúja.}{主已示亞勒路亞,厥救世者亞勒路亞。}
\psPsalmTitle{Psalmus 97}{聖詠97}
\psVerse{Cantáte Dómino cánticum novum: * \\quia mirabília fecit.}{請爾等向主奏新歌,\\盖厥行奇異。}
\psVerse{Salvávit sibi déxtera ejus: * \\et bráchium sanctum ejus.}{厥右手與膊救伊。}
\psVerse{Notum fecit Dóminus salutáre suum: * \\in conspéctu géntium revelávit justítiam suam.}{主已顯救世者,\\異教之前顯明厥義。}
\psVerse{Recordátus est misericórdiæ suæ, * \\et veritátis suæ dómui Israël.}{記念厥慈及其真,\\為依撒厄爾家。}
\psVerse{Vidérunt omnes términi terræ * \\salutáre Dei nostri.}{遍地末境\\已覩救世我等天主。}
\psVerse{Jubiláte Deo, omnis terra: * \\cantáte, et exsultáte, et psállite.}{請遍地向主歡慶,\\且歌且咏且抃舞。}
\psVerse{Psállite Dómino in cíthara, in cíthara et voce psalmi: * \\in tubis ductílibus, et voce tubæ córneæ.}{向主弹咏二十四弦琴,弹咏二十四弦琴,\\用金號與角號,吹主前。}
\psVerse{Jubiláte in conspéctu regis Dómini: * \\moveátur mare, et plenitúdo ejus: orbis terrárum, et qui hábitant in eo.}{熙熙寰海與其所包涵之物,\\寰宇與其所居者亦欣動。}
\psVerse{Flúmina plaudent manu, simul montes exsultábunt a conspéctu Dómini: * \\quóniam venit judicáre terram.}{江河擺手,並山岳主臺前踴躍。\\盖降臨審判遍地,}
\psVerse{Judicábit orbem terrárum in justítia, * \\et pópulos in æquitáte.}{將以義判天下,\\以均平萬民。}
\psGloria{Glória Patri, et Fílio, * et Spirítui Sancto. \\Sicut erat in princípio, et nunc, et semper, * et in sǽcula sæculórum. Amen.}{欽頌榮福於罷德肋、及費畧、及斯彼利多三多,\\若今茲、若永遠、及無窮世。亞孟。}
\psAntiphonRepeat{Notum fecit Dóminus, Allelúja, salutáre suum, Allelúja.}{主已示亞勒路亞,厥救世者亞勒路亞。}
\psVR{V}{Ipse invocábit me, allelúja.}{伊將呼予亞勒路亞。}
\psVR{R}{Pater meus es tu, allelúja.}{爾乃予父亞勒路亞。}
\psRubric{Pater noster (totum dicitur secreto)}{默念天主經 (在天云云)}
\psVR{V}{Et ne nos indúcas in tentatiónem.}{而不我許陷於誘感。}
\psVR{R}{Sed líbera nos a malo.}{乃救我於凶惡。}
\psRubric{Absolutio}{畢經}
\psCollect{A vínculis peccatórum nostrórum absólvat nos omnípotens et miséricors Dóminus.}{望全能仁慈天主釋我等於罪桎梏。}
\psVR{R}{Amen.}{亞孟。}
\psRubric{Benedictio}{降福}
\psVR{V}{Jube, domne, benedícere.}{請降福。}
\psCollect{Evangélica léctio sit nobis salus et protéctio.}{萬日畧經護衛我等。}
\psVR{R}{Amen.}{亞孟。}
\psLesson{Lectio VII: Initium S. Evangelii sec. Lucam}{書七 (依聖路嘉萬日畧經)}
\psText{In illo témpore: Exiit edíctum a Cǽsare Augústo, ut describerétur univérsus orbis. Et réliqua.}{維時責撒肋奥吾斯多發令,命厥攸屬邦人,皆報名於官。}
\psLesson{Homilía sancti Gregórii Papæ}{聖額我略教皇經解}
\psText{Quia, largiénte Dómino, Missárum solémnia ter hódie celebratúri sumus, loqui diu de Evangélio non pósumus. Sed nos aliquid vel bréviter dícere, Redemptóris nostri nativítas ipsa compéllit. Quid est enim, quod nascitúro Dómino mundus describítur, nisi hoc, quod apérte monstrátur, quia ille apparébat in carne, qui electórum suórum númerum adscribéret in æternitáte?}{今日蒙主賜獻祭三次,無能申解萬日畧經,畧述大旨。主將誕,衆名報籍何義?示主降世,俾攸選者錄天冊也。先知者云:被逐者即塗於生活者之籍,不與義人同冊焉。主誕於白稜郡者何?白稜譯麫餅舍。主自云:予乃由天降活餅。誕處謂麫餅之舍,盖主取形體為人,將見於彼,飽飫諸聖之心於天國矣。誕不於家而於途,猶示取人性而生不在本處。}
\psVR{R}{Beáta víscera Maríæ Vírginis, quæ portavérunt ætérni Patris Fílium: et beáta úbera, quæ lactavérunt Christum Dóminum: * \\Quia hódie pro salúte mundi de Vírgine nasci dignátus est.}{福哉聖母胎懷,懷永父之子。福乳哉,哺主基利斯督。 * \\伊今日為救世,不棄遺生於童女。}
\psVR{V}{Dies sanctificátus illúxit nobis: veníte, gentes, et adoráte Dóminum.}{聖日照吾儕,異教者請欽崇主。}
\psRubric{Repetitur}{重念}
\psText{Quia hódie pro salúte mundi de Vírgine nasci dignátus est.}{伊今日為救世。}
\psRubric{Benedictio}{降福}
\psVR{V}{Jube, domne, benedícere.}{請降福。}
\psCollect{Divínum auxílium máneat semper nobíscum.}{願天主之祐常偕我等。}
\psVR{R}{Amen.}{亞孟。}
\psLesson{Lectio VIII: Initium S. Evangelii sec. Lucam}{書八 (依聖路嘉萬日畧經)}
\psText{In illo témpore: Pastóres loquebántur ad ínvicem: Transeámus usque Béthlehem, et videámus hoc verbum, quod factum est, quod Dóminus osténdit nobis. Et réliqua.}{維時牧童胥言:徃也,亟至白稜,徃視天主今攸為奇,而示于吾。}
\psLesson{Homilía sancti Ambrósii Epíscopi}{聖盎博羅削主教經解}
\psText{Vidéte inítium surgéntis Ecclésiæ. Christus náscitur, et pastóres vigiláre cœpérunt: ut géntium greges, pecudum modo vivéntes, in caulam Dómini congregárent, ne quos spirituálium bestiárum, in noctis ténebris, pateréntur incúrsus. Et bene pastóres vígilant, quos bonus Pastor infórmat. Grex est pópulus, nox est sǽculum, pastóres sunt sacerdótes.}{試觀聖教初行,主生而牧童守夜。異教者居世,如俾牲歸聚於主棧,免中夜異獸之害。善收者訓牧童守夜,豈不宜哉?綿羊其衆民,夜其此世,收童其撒責爾鐸德也。或云經彼云謹守堅固,盖天主在世,不但令主教者,並遣天神護羊也。}
\psVR{R}{Verbum caro factum est, et habitávit in nobis: * \\Et vídimus glóriam ejus, glóriam quasi Unigéniti a Patre, plenum grátiæ et veritátis.}{物爾朋已降為人,已居吾內。 * \\吾已覲厥榮光,儼然吾聖真子,厥靈充盈以聖寵以真實。}
\psVR{V}{Omnia per ipsum facta sunt, et sine ipso factum est nihil.}{萬物由之造,匪斯無一物。}
\psRubric{Repetitur}{重念}
\psText{Et vídimus glóriam ejus, glóriam quasi Unigéniti a Patre, plenum grátiæ et veritátis.}{吾已覲厥榮光,儼然吾聖真子,厥靈充盈以聖寵以真實。}
\psGloria{Glória Patri, et Fílio, et Spirítui Sancto.}{欽頌榮福於罷德肋、及費畧、及斯彼利多三多。}
\psRubric{Repetitur}{重念}
\psText{Et vídimus glóriam ejus, glóriam quasi Unigéniti a Patre, plenum grátiæ et veritátis.}{吾已覲厥榮光,儼然吾聖真子,厥靈充盈以聖寵以真實。}
\psRubric{Benedictio}{降福}
\psVR{V}{Jube, domne, benedícere.}{請降福。}
\psCollect{Ad societátem cívium supernórum perdúcat nos Rex Angelórum.}{天神之王,引我等至於天上諸聖之列。}
\psVR{R}{Amen.}{亞孟。}
\psLesson{Lectio IX: Initium S. Evangelii sec. Joannem}{書九 (依聖若望萬日畧經)}
\psText{In princípio erat Verbum, et Verbum erat apud Deum, et Deus erat Verbum. Et réliqua.}{厥始物爾朋已有,斯物爾朋實在天主,實即天主。}
\psLesson{Homilía sancti Augustíni Epíscopi}{聖奥吾斯定主教經解}
\psText{Quod quæris quid sit Verbum? Non arbitror te hominem esse, et non intellegere verbum hominis. Et tamen audis verbum hominis, et intellegis verbum Dei? Audis verbum quod non est Deus, et intellegis verbum quod est Deus? Si Deus creavit verbum, si Deus genuit verbum? Si Deus creavit verbum, non est Deus, si Deus genuit verbum, Deus est.}{凡問天主乃言,勿意與世之謂言同義,陋哉此意也。今黨亞利阿者有曰:天主言受造。既天主言受造,則經何謂萬物由言受造?若云天主言受造,試問於何言受造?若云此乃言之言,由彼受造,斯即予謂惟一天主子。若不云言之言,則宜云不受造也。盖因彼萬物受造,則不能自造,是故宜信聖史者。}
\psRubric{Te Deum}{盎博羅削奧吾斯定兩位聖師共同聖詠}
\psText{Te Deum laudámus: * \\te Dóminum confitémur.}{吾侪讚頌天主, * \\認為真主。}
\psText{Te ætérnum Patrem * \\omnis terra venerátur.}{遍地皆欽崇 * \\無始之父。}
\psText{Tibi omnes Angeli, * \\tibi Cæli, et univérsæ Potestátes:}{諸神諸天 * \\諸德,}
\psText{Tibi Chérubim et Séraphim * \\incessábili voce proclámant:}{並上智者熾愛者天神, * \\不断呼號曰:}
\psText{Sanctus, Sanctus, Sanctus * \\Dóminus Deus Sábaoth.}{聖、聖、聖, * \\撒巴阿德天主。}
\psText{Pleni sunt cæli et terra * \\majestátis glóriæ tuæ.}{天地滿被 * \\巍顯之榮。}
\psText{Te gloriósus * \\Apostolórum chorus,}{宗徒之 * \\榮位,}
\psText{Te Prophetárum * \\laudábilis númerus,}{先知之 * \\羣衆,}
\psText{Te Mártyrum candidátus * \\laudat exércitus.}{致命之軍旅, * \\皆讚頌主。}
\psText{Te per orbem terrárum * \\sancta confitétur Ecclésia,}{遍地聖教 * \\公認主,}
\psText{Patrem * \\imménsæ majestátis;}{為無疆 * \\威嚴之父。}
\psText{Venerándum tuum verum * \\et únicum Fílium;}{並聖子當欽崇, * \\且真且惟一。}
\psText{Sanctum quoque * \\Paráclitum Spíritum.}{並安慰 * \\之聖神。}
\psText{Tu Rex glóriæ, * \\Christe.}{榮福之王 * \\基利斯督,}
\psText{Tu Patris * \\sempitérnus es Fílius.}{為聖父 * \\無始之子,}
\psText{Tu, ad liberándum susceptúrus hóminem, * \\non horruísti Vírginis úterum.}{為救贖人類 * \\降童貞之胎。}
\psText{Tu, devícto mortis acúleo, * \\aperuísti credéntibus regna cælórum.}{擊敗永死, * \\為諸信者啟天門。}
\psText{Tu ad déxteram Dei sedes, * \\in glória Patris.}{坐天主右, * \\偕父同榮。}
\psText{Judex créderis * \\esse ventúrus.}{我等信 * \\審判生死者。}
\psRubric{Genuflectitur}{(跪)}
\psText{Te ergo quǽsumus, tuis fámulis súbveni, * \\quos pretióso sánguine redemísti.}{今懇扶祐諸僕, * \\當時以寶血所救贖者,}
\psRubric{Levatur}{(起)}
\psText{Ætérna fac cum Sanctis tuis * \\in glória numerári.}{使偕諸聖 * \\享受榮福之報。}
\psText{Salvum fac pópulum tuum, Dómine, * \\et bénedic hereditáti tuæ.}{望主拯救厥民, * \\降福於厥嗣業,}
\psText{Et rege eos, * \\et extólle illos usque in ætérnum.}{政治伊等, * \\而簡擢於無窮世。}
\psText{Per síngulos dies * \\benedicimus te.}{每日吾儕 * \\當讚頌主,}
\psText{Et laudámus nomen tuum in sǽculum, * \\et in sǽculum sǽculi.}{讚頌主名 * \\於無疆永世。}
\psText{Dignáre, Dómine, die isto * \\sine peccáto nos custodíre.}{懇賜吾歷今日 * \\無罪,}
\psText{Miserére nostri, Dómine, * \\miserére nostri.}{主矜憐我等, * \\主矜憐我等。}
\psText{Fiat misericórdia tua, Dómine, super nos, * \\quemádmodum sperávimus in te.}{如吾所望, * \\降仁慈於我等,}
\psText{In te, Dómine, sperávi: * \\non confúndar in ætérnum.}{吾藉主 * \\永不負屈。}
\psRubric{Conclusio}{畢經}
\psVR{V}{Dóminus vobíscum.}{主與爾偕。}
\psVR{R}{Et cum spíritu tuo.}{並於爾神。}
\psRubric{Oratio}{祝文}
\psCollect{Concéde, quǽsumus, omnípotens Deus: ut nos Unigéniti tui nova per carnem Natívitas líberet; quos sub peccáti jugo vetústa sérvitus tenet. Per eúmdem Dóminum nostrum Jesum Christum Fílium tuum: Qui tecum vivit et regnat in unitáte Spíritus Sancti Deus, per ómnia sǽcula sæculórum.}{懇祈全能天主,彼久在罪軛,賴爾惟一子肉身新誕,幸救脫亦。為耶穌基利斯督爾子我等主,其偕爾偕斯彼利多三多,為一天主,永生永王。}
\psVR{R}{Amen.}{亞孟。}
\psVR{V}{Dóminus vobíscum.}{主與爾偕。}
\psVR{R}{Et cum spíritu tuo.}{並於爾神。}
\psVR{V}{Benedicámus Dómino.}{讚美主。}
\psVR{R}{Deo grátias.}{謝天主。}
\psVR{V}{Fidélium ánimæ per misericórdiam Dei requiéscant in pace.}{凡諸信者靈魂,賴天主仁慈息止安所。}
\psVR{R}{Amen.}{亞孟。}
\psThickRule
\psHeaderOneLowercase{Ad Laudes}{讚美經}
\psRubric{(Dícitur Pater noster et Ave María, secréto.)}{默念:天主經、聖母經。\\ (畢,明聲念:)}
\psVR{V}{Deus, in adiutórium meum inténde.}{天主惟專於我扶祐。}
\psVR{R}{Dómine, ad adjuvándum me festína.}{主速格以救助我。}
\psGloria{Glória Patri, et Fílio, et Spirítui Sancto. Sicut erat in princípio, et nunc, et semper, et in sǽcula sæculórum. Amen. Allelúja.}{欽頌榮福於罷德肋、及費畧、及斯彼利多三多,\\若今茲、若永遠、及無窮世。亞孟。亞勒路亞。}
\psAntiphonRepeat{Quem vidístis, pastóres? dícite, annuntiáte nobis, quis appáruit in terris? Natum vídimus, et choros Angelórum collaudántes Dóminum, allelúja, allelúja.}{牧童奚見?請示我等見世為疇?吾見已生者,並衆天神合讚美主。亞勒路亞,亞勒路亞。}
\psPsalmTitle{Psalmus 92}{聖詠92}
\psVerse{Dóminus regnávit, decórem indútus est: * \\indútus est Dóminus fortitúdinem, et præcínxit se.}{主已王矣,衣美麗,\\衣勇德而自繫。}
\psVerse{Étenim firmávit orbem terræ, * \\qui non commovébitur.}{盖堅定宇宙,\\將不易動。}
\psVerse{Paráta sedes tua ex tunc: * \\a sǽculo tu es.}{厥寶座自彼時備,\\厥自無始有主。}
\psVerse{Elevavérunt flúmina, Dómine: * \\elevavérunt flúmina vocem suam.}{江河已舉,\\江河已舉厥聲。}
\psVerse{Elevavérunt flúmina fluctus suos, * \\a vócibus aquárum multárum.}{江河舉浪,\\由多水之衆聲。}
\psVerse{Mirábiles elatiónes maris: * \\mirábilis in altis Dóminus.}{異哉海湧,\\主於諸高峻。}
\psVerse{Testimónia tua credibília facta sunt nimis: * \\domum tuam decet sanctitúdo, Dómine, in longitúdinem diérum.}{異哉諸証,最為信真。\\厥宮宜為聖於無窮世。}
\psGloria{Glória Patri, et Fílio, * et Spirítui Sancto. \\Sicut erat in princípio, et nunc, et semper, * et in sǽcula sæculórum. Amen.}{欽頌榮福於罷德肋、及費畧、及斯彼利多三多,\\若今茲、若永遠、及無窮世。亞孟。}
\psAntiphonRepeat{Quem vidístis, pastóres? dícite, annuntiáte nobis, quis appáruit in terris? Natum vídimus, et choros Angelórum collaudántes Dóminum, allelúja, allelúja.}{牧童奚見?請示我等見世為疇?吾見已生者,並衆天神合讚美主。亞勒路亞,亞勒路亞。}
\psAntiphonRepeat{Génuit puérpera Regem, cui nomen ætérnum, et gáudia matris habens cum virginitátis honóre: nec primam símilem visa est, nec habére sequéntem, allelúja.}{童女產王,厥名惟永享母皇,並童貞之榮,尊推於前,引於後,無可擬者。亞勒路亞。}
\psPsalmTitle{Psalmus 99}{聖詠99}
\psVerse{Jubiláte Deo, omnis terra: * \\servíte Dómino in lætítia.}{遍地宜踴躍而歡慶,\\奉事主。}
\psVerse{Introíte in conspéctu ejus, * \\in exsultatióne.}{請踴躍進臺前,\\(宜知主真天主)。}
\psVerse{Scitóte quóniam Dóminus ipse est Deus: * \\ipse fecit nos, et non ipsi nos.}{宜知主真天主。\\造我等,非我等自造。}
\psVerse{Pópulus ejus, et oves páscuæ ejus: * \\introíte portas ejus in confessióne, átria ejus in hymnis: confitémini illi.}{厥民且所牧之羊。\\極頌而進厥門,至於廊廡。奏聖詠而稱頌之,}
\psVerse{Laudáte nomen ejus: quóniam suávis est Dóminus, in ætérnum misericórdia ejus, * \\et usque in generatiónem et generatiónem véritas ejus.}{稱頌主名。盖主怡哉,厥慈永永,\\而厥真實無窮世。}
\psGloria{Glória Patri, et Fílio, * et Spirítui Sancto. \\Sicut erat in princípio, et nunc, et semper, * et in sǽcula sæculórum. Amen.}{欽頌榮福於罷德肋、及費畧、及斯彼利多三多,\\若今茲、若永遠、及無窮世。亞孟。}
\psAntiphonRepeat{Génuit puérpera Regem, cui nomen ætérnum, et gáudia matris habens cum virginitátis honóre: nec primam símilem visa est, nec habére sequéntem, allelúja.}{童女產王,厥名惟永享母皇,並童貞之榮,尊推於前,引於後,無可擬者。亞勒路亞。}
\psAntiphonRepeat{Angelus ad pastóres ait: Annúntio vobis gáudium magnum: quia natus est vobis hódie Salvátor mundi, allelúja.}{天神向牧童曰:予來報爾莫大喜音,救世主為爾適誕。亞勒路亞。}
\psPsalmTitle{Psalmus 62}{聖詠62}
\psVerse{Deus, Deus meus, * \\ad te de luce vígilo.}{天主吾天主,\\予晨寤向主臺前。}
\psVerse{Sitívit in te ánima mea, * \\quam multiplíciter tibi caro mea.}{予神渴慕主,\\舉躬望主屢屢哉。}
\psVerse{In terra desérta, et ínvia, et inaquósa: * \\sic in sancto appárui tibi, ut vidérem virtútem tuam, et glóriam tuam.}{於野地無人迹無水澤,\\予現主前如於聖殿然,以覩厥德義與榮福。}
\psVerse{Quóniam mélior est misericórdia tua super vitas: * \\lábia mea laudábunt te.}{盖主寵慈更超於諸生命之美,\\吾口將讚頌主。}
\psVerse{Sic benedícam te in vita mea: * \\et in nómine tuo levábo manus meas.}{於是生時讚頌主,\\而因主名舉揚予手。}
\psVerse{Sicut ádipe et pinguédine repleátur ánima mea: * \\et lábiis exsultatiónis laudábit os meum.}{予靈如膏脂充滿,\\以口躍然讚頌。}
\psVerse{Si memor fui tui super stratum meum, in matutínis meditábor in te: * \\quia fuísti adjútor meus.}{寢則記憶主,晨寤而默想主。\\盖主寵祐及之,}
\psVerse{Et in velaménto alárum tuárum exsultábo, adhǽsit ánima mea post te: * \\me suscépit déxtera tua.}{予在翼下當欣躍,而靈密邇隨主。\\主右手接之,}
\psVerse{Ipsi vero in vanum quæsiérunt ánimam meam, introíbunt in inferióra terræ: * \\tradéntur in manus gládii, partes vúlpium erunt.}{彼輩徒然謀害予。將入地心,\\付於刀劍之手,為狐狸之一分。}
\psVerse{Rex vero lætábitur in Deo, laudabúntur omnes qui jurant in eo: * \\quia obstrúctum est os loquéntium iníqua.}{王將享樂於天主。凡稱天主名矢向彼者,應被褒許。\\盖言不義者之口塞矣。}
\psGloria{Glória Patri, et Fílio, * et Spirítui Sancto. \\Sicut erat in princípio, et nunc, et semper, * et in sǽcula sæculórum. Amen.}{欽頌榮福於罷德肋、及費畧、及斯彼利多三多,\\若今茲、若永遠、及無窮世。亞孟。}
\psAntiphonRepeat{Angelus ad pastóres ait: Annúntio vobis gáudium magnum: quia natus est vobis hódie Salvátor mundi, allelúja.}{天神向牧童曰:予來報爾莫大喜音,救世主為爾適誕。亞勒路亞。}
\psAntiphonRepeat{Facta est cum Angelo multitúdo cæléstis exércitus laudántium Deum, et dicéntium: Glória in excélsis Deo, et in terra pax homínibus bonæ voluntátis, allelúja.}{多神俄見頌聲滿空,曰:天主受享榮福于天,良善受享太平于地。亞勒路亞。}
\psPsalmTitle{Canticum Trium Puerorum (Dan. 3)}{三童歌}
\psVerse{Benedícite, ómnia ópera Dómini, Dómino: * \\laudáte et superexaltáte eum in sǽcula.}{天主諸受造者,請讚美主,\\稱頌舉揚於無窮世。}
\psVerse{Benedícite, Angeli Dómini, Dómino: * \\benedícite, cæli, Dómino.}{天主神使請讚美主,\\在上諸天請讚美主。}
\psVerse{Benedícite, aquæ omnes, quæ super cælos sunt, Dómino: * \\benedícite, omnes virtútes Dómini, Dómino.}{天上諸水俱讚美主,\\諸天之德俱讚美主。}
\psVerse{Benedícite, sol et luna, Dómino: * \\benedícite, stellæ cæli, Dómino.}{太陽太陰俱讚美主,\\星辰宿度俱讚美主。}
\psVerse{Benedícite, omnis imber et ros, Dómino: * \\benedícite, omnes spíritus Dei, Dómino.}{空中雨露俱讚美主,\\諸品天神俱讚美主。}
\psVerse{Benedícite, ignis et æstus, Dómino: * \\benedícite, frigus et æstus, Dómino.}{火也暑也俱讚美主,\\寒也熱也俱讚美主。}
\psVerse{Benedícite, rores et pruína, Dómino: * \\benedícite, gelu et frigus, Dómino.}{露也霜也俱讚美主,\\凍也冷也俱讚美主。}
\psVerse{Benedícite, glácies et nives, Dómino: * \\benedícite, noctes et dies, Dómino.}{氷也雪也俱讚美主,\\晝也夜也俱讚美主。}
\psVerse{Benedícite, lux et ténebræ, Dómino: * \\benedícite, fúlgura et nubes, Dómino.}{光也暗也俱讚美主,\\閃電繁雲俱讚美主。}
\psVerse{Benedícat terra Dóminum: * \\laudet et superexaltét eum in sǽcula.}{廣輿全球請讚美主,\\稱頌舉揚於無窮世。}
\psVerse{Benedícite, montes et colles, Dómino: * \\benedícite, univérsa germinántia in terra, Dómino.}{山岳丘陵俱讚美主,\\坤載廣生俱讚美主。}
\psVerse{Benedícite, fontes, Dómino: * \\benedícite, mária et flúmina, Dómino.}{流泉衆派俱讚美主,\\環海江河俱讚美主。}
\psVerse{Benedícite, cete, et ómnia, quæ movéntur in aquis, Dómino: * \\benedícite, omnes vólucres cæli, Dómino.}{巨鰓細鱗俱讚美主,\\空中羽毛俱讚美主。}
\psVerse{Benedícite, omnes béstiæ et pécora, Dómino: * \\benedícite, fílii hóminum, Dómino.}{畜牲禽獸俱讚美主,\\黔首蒼生俱讚美主。}
\psVerse{Benedícat Israël Dóminum: * \\laudet et superexaltét eum in sǽcula.}{義撒國人俱讚美主,\\稱頌舉揚於無窮世。}
\psVerse{Benedícite, sacerdótes Dómini, Dómino: * \\benedícite, servi Dómini, Dómino.}{主之鐸德俱讚美主,\\主之役輩俱讚美主。}
\psVerse{Benedícite, spíritus, et ánimæ justórum, Dómino: * \\benedícite, sancti, et húmiles corde, Dómino.}{義人之靈神俱讚美主,\\諸聖及謙德者俱讚美主。}
\psVerse{Benedícite, Ananía, Azaría, Mísaël, Dómino: * \\laudáte et superexaltáte eum in sǽcula.}{亞納尼亞、啞理亞、彌颯厄爾俱讚美主,\\稱頌舉揚於無窮世。}
\psVerse{Benedicámus Patrem et Fílium cum Sancto Spíritu: * \\laudémus et superexaltémus eum in sǽcula.}{請衆同讚父子聖神,\\稱頌舉揚於無窮世。}
\psVerse{Benedíctus es, Dómine, in firmaménto cæli: * \\et laudábilis, et gloriósus, et superexaltátus in sǽcula.}{主於穩定天為殊福,及當讚美,\\及光榮舉揚於無窮世。}
\psRubric{(Hic non dicitur Gloria Patri.)}{(此處不念欽頌榮福)}
\psAntiphonRepeat{Facta est cum Angelo multitúdo cæléstis exércitus laudántium Deum, et dicéntium: Glória in excélsis Deo, et in terra pax homínibus bonæ voluntátis, allelúja.}{多神俄見頌聲滿空,曰:天主受享榮福于天,良善受享太平于地。亞勒路亞。}
\psAntiphonRepeat{Párvulus fílius hódie natus est nobis: et vocábitur Deus, Fortis, allelúja.}{今日嬰孩為吾生誕,將稱勇天主。亞勒路亞。}
\psPsalmTitle{Psalmus 148}{聖詠148}
\psVerse{Laudáte Dóminum de cælis: * \\laudáte eum in excélsis.}{請衆自諸天讚美主,\\而於峻極讚美之。}
\psVerse{Laudáte eum, omnes Angeli ejus: * \\laudáte eum, omnes virtútes ejus.}{衆天神讚美主,\\厥諸品德讚美主。}
\psVerse{Laudáte eum, sol et luna: * \\laudáte eum, omnes stellæ et lumen.}{太陽太陰讚美主,\\諸星與光讚美主。}
\psVerse{Laudáte eum, cæli cælórum: * \\et aquæ omnes, quæ super cælos sunt, laudent nomen Dómini.}{諸天之天讚美主,\\及在諸天上之水讚美主名。}
\psVerse{Quia ipse dixit, et facta sunt: * \\ipse mandávit, et creáta sunt.}{盖主一言而萬有受成,\\主一命而萬有受造。}
\psVerse{Státuit ea in ætérnum, et in sǽculum sǽculi: * \\præcéptum pósuit, et non præteríbit.}{建定列品永永無窮世,\\施令而不越。}
\psVerse{Laudáte Dóminum de terra, * \\dracónes, et omnes abyssi.}{請讚主於地:\\蛟龍及諸淵,}
\psVerse{Ignis, grando, nix, glácies, spíritus procellárum: * \\quæ fáciunt verbum ejus:}{火雹雪氷諸風波之能,\\凡聽厥命者。}
\psVerse{Montes, et omnes colles: * \\ligna fructífera, et omnes cedri.}{山岳與諸谷,\\結實之木與凡則獨鹿樹。}
\psVerse{Béstiæ, et univérsa pécora: * \\serpéntes, et vólucres pennátæ:}{禽獸與諸畜牲,\\迤行與衆羽。}
\psVerse{Reges terræ, et omnes pópuli: * \\príncipes, et omnes júlices terræ.}{世王與蒼生長帥,\\及凡聽讞者。}
\psVerse{Júvenes, et vírgines: senes cum junióribus laudent nomen Dómini: * \\quia exaltátum est nomen ejus solíus.}{青年與衆童女,耄耋與幼稚者。請頌主名。\\盖獨一主名巍巍,}
\psVerse{Conféssio ejus super cælum et terram: * \\et exaltávit cornu pópuli sui.}{聲施上天下地。\\而主亦顯厥民之榮,}
\psVerse{Hymnus ómnibus sanctis ejus: * \\fíliis Israël, pópulo appropinquánti sibi.}{主之諸聖宜咏頌,\\義撒厄爾衆子及近主之民。}
\psGloria{Glória Patri, et Fílio, * et Spirítui Sancto. \\Sicut erat in princípio, et nunc, et semper, * et in sǽcula sæculórum. Amen.}{欽頌榮福於罷德肋、及費畧、及斯彼利多三多,\\若今茲、若永遠、及無窮世。亞孟。}
\psAntiphonRepeat{Párvulus fílius hódie natus est nobis: et vocábitur Deus, Fortis, allelúja.}{今日嬰孩為吾生誕,將稱勇天主。亞勒路亞。}
\psRubric{Capitulum}{節目}
\psHeaderThree{Hebr. 1:1-2}{希一1,2}
\psText{Multifáriam, multisque modis olim Deus loquens pátribus in prophétis: novíssime diébus istis locútus est nobis in Fílio, quem constítuit heréde universórum, per quem fecit et sǽcula.}{昔天主屢次多方,啟先知者轉語吾祖;末時斯日,由厥子嗣萬有造寰宇者,語我等。}
\psVR{R}{Deo grátias.}{謝天主。}
\psRubric{Hymnus}{聖歌}
\psHymnHeader{A solis ortus cárdine}{自日出樞至地末境}
\psHymnStanza{A solis ortus cárdine Adúsque terræ límitem, \\Christum canámus Príncipem, Natum María Vírgine.}{自日出樞至地末境,稱頌吾主。\\由童女生,肇造公主披微僕軀。}
\psHymnStanza{Beátus Auctor sǽculi Servíle corpus índuit: \\Ut carne carnem líberans, Ne pérderet quos cóndidit.}{由形救形,攸造免淪。}
\psHymnStanza{Castæ Paréntis víscera Cæléstis intrat grátia: \\Venter puéllæ bájulat Secréta, quæ non nóverat.}{聖寵降臨,童女胞內;\\貞女懷抱,難測奧義。}
\psHymnStanza{Domus pudíci péctoris Templum repénte fit Dei: \\Intácta nésciens virum, Verbo concépit Fílium.}{貞心厥室,倏成主殿,\\不喻厥男,懷娠聖子。}
\psHymnStanza{Gaudet chorus cæléstium, Et Ángeli canunt Deo; \\Palamque fit pastóribus Pastor, Creátor ómnium.}{天神預示,若望覺知;\\母胎踴躍,童女誕育。}
\psHymnStanza{Eníxa est puérpera, Quem Gábriel prædíxerat: \\Quem Matris alvo géstiens, Clausus Joánnes sénserat.}{置身草茅,不棄馬櫪。\\養育蒼生,以微乳哺。}
\psHymnStanza{Fœno jacére pértulit, Præsépe non exhórruit: \\Et lacte módico pastus est, Per quem nec ales ésurit.}{上天榮福,天神恭慶;\\拯世慈主,見身牧童。}
\psHymnStanza{Jesu, tibi sit glória, Qui natus es de Vírgine, \\Cum Patre et almo Spíritu, In sempitérna sǽcula. Amen.}{生於童女耶穌,永福歸父偕子,\\聖神世世。亞孟。}
\psVR{V}{Notum fecit Dóminus, allelúja.}{主已顯亞勒路亞。}
\psVR{R}{Salutáre suum, allelúja.}{厥救世者亞勒路亞。}
\psRubric{Ad Benedictus}{匝加利亞聖歌(聖母歌)}
\psAntiphonRepeat{Glória in excélsis Deo, et in terra pax homínibus bonæ voluntátis, allelúja.}{天主受享榮福于天,良善受享太平于地。亞勒路亞。}
\psPsalmTitle{Canticum Zachariæ (Luc. 1:68-79)}{匝加利亞聖歌}
\psVerse{Benedíctus Dóminus, Deus Israël, * \\quia visitávit, et fecit redemptiónem plebis suæ:}{頌謝吾主義撒厄爾天主,\\眷顧救贖厥民。}
\psVerse{Et eréxit cornu salútis nobis: * \\in domo David, púeri sui.}{起而勇救我,\\於達味德厥僕之家。}
\psVerse{Sicut locútus est per os sanctórum, * \\qui a sǽculo sunt, prophetárum ejus:}{踐逆知者古聖口諭,}
\psVerse{Salútem ex inimícis nostris, * \\et de manu ómnium, qui odérunt nos:}{救脫於我仇,\\及凡憎害我者。}
\psVerse{Ad faciéndam misericórdiam cum pátribus nostris: * \\et memorári testaménti sui sancti.}{施仁慈於吾祖,\\而記憶厥詔。}
\psVerse{Jusjurándum, quod jurávit ad Ábraham patrem nostrum, * \\datúrum se nobis:}{向亞巴郎吾祖矢,\\將賜與我等。}
\psVerse{Ut sine timóre, de manu inimicórum nostrórum liberáti, * \\serviámus illi.}{脫吾仇之手,使無懼生平,}
\psVerse{In sanctitáte, et justítia coram ipso, * \\ómnibus diébus nostris.}{於主臺前以義德以聖德\\而奉事之。}
\psVerse{Et tu, puer, Prophéta Altíssimi vocáberis: * \\præíbis enim ante fáciem Dómini, paráre vias ejus:}{幼童爾將稱天主前知者,\\盖主前將驅治厥道。}
\psVerse{Ad dandam sciéntiam salútis plebi ejus: * \\in remissiónem peccatórum eórum:}{令厥民得知其救贖,\\赦免其罪。}
\psVerse{Per víscera misericórdiæ Dei nostri: * \\in quibus visitávit nos, óriens ex alto:}{因吾天主仁慈之心,\\從上而出俯顧我衆。}
\psVerse{Illumináre his, qui in ténebris, et in umbra mortis sedent: * \\ad dirigéndos pedes nostros in viam pacis.}{照牖凡居暗處及於死影,\\導引吾足行平和之道。}
\psGloria{Glória Patri, et Fílio, * et Spirítui Sancto. \\Sicut erat in princípio, et nunc, et semper, * et in sǽcula sæculórum. Amen.}{欽頌榮福於罷德肋、及費畧、及斯彼利多三多,\\若今茲、若永遠、及無窮世。亞孟。}
\psAntiphonRepeat{Glória in excélsis Deo, et in terra pax homínibus bonæ voluntátis, allelúja.}{天主受享榮福于天,良善受享太平于地。亞勒路亞。}
\psRubric{Oratio}{祝文}
\psCollect{Concéde, quǽsumus, omnípotens Deus: ut nos Unigéniti tui nova per carnem Natívitas líberet; quos sub peccáti jugo vetústa sérvitus tenet. Per eúndem Dóminum nostrum Jesum Christum Filium tuum: Qui tecum vivit et regnat in unitate Spiritus Sancti Deus, per omnia saecula saeculorum.}{懇祈全能天主,彼久在罪軛,賴爾惟一子肉身新誕,幸救脫亦。為聖子耶穌基利斯督我等主,其偕爾偕斯彼利多三多,為一天主,永生永王。}
\psVR{R}{Amen.}{亞孟。}
\psVR{V}{Dóminus vobíscum.}{主與爾偕。}
\psVR{R}{Et cum spíritu tuo.}{並於爾神。}
\psVR{V}{Benedicámus Dómino.}{讚美主。}
\psVR{R}{Deo grátias.}{謝天主。}
\psVR{V}{Fidélium ánimæ per misericórdiam Dei requiéscant in pace.}{凡諸信者靈魂,賴天主仁慈息止安所。}
\psVR{R}{Amen.}{亞孟。}
\psRubric{(Pater noster, secreto.)}{(後默念天主經)}
\psThickRule
\psHeaderOneLowercase{Ad Primam}{一時經 (晨經)}
\psRubric{(Dícitur Pater noster, Ave María, et Credo, secréto.)}{默念:天主經、聖母經、信經。\\ (畢,明聲念:)}
\psVR{V}{Deus, in adiutórium meum inténde.}{天主惟專於我扶祐。}
\psVR{R}{Dómine, ad adjuvándum me festína.}{主速格以救助我。}
\psGloria{Glória Patri, et Fílio, et Spirítui Sancto. Sicut erat in princípio, et nunc, et semper, et in sǽcula sæculórum. Amen. Allelúja.}{欽頌榮福於罷德肋、及費畧、及斯彼利多三多,\\若今茲、若永遠、及無窮世。亞孟。亞勒路亞。}
\psRubric{Hymnus}{聖歌}
\psHymnHeader{Jam lucis orto sídere}{啟明已升}
\psHymnStanza{Jam lucis orto sídere, Deum precémur súpplices: \\Ut in diúrnis áctibus, Nos servet a nocéntibus.}{啟明已升,懇祈天主,\\竟日行為,護吾逆害。}
\psHymnStanza{Linguam refrénans témperet, Ne litis horror ínsonet: \\Visum fovéndo cóntegat, Ne vanitátes háuriat.}{馴其口勿啟譁競,\\謹守瞳輪勿涉虛妄。}
\psHymnStanza{Sint pura cordis íntima, Absístat et vecórdia: \\Carnis terat supérbiam Potus cibíque párcitas.}{心宮潔清,勿生放怠。\\淡食薄飲,消制氣傲。}
\psHymnStanza{Ut, cum dies abscésserit, Noctémque sors redúxerit, \\Mundi per abstinéntiam Ipsi canámus glóriam.}{俟日光退,夜暗繼之,\\全守嚴規,稱謝主恩。}
\psHymnStanza{Jesu, tibi sit glória, Qui natus es de Vírgine, \\Cum Patre et almo Spíritu, In sempitérna sǽcula. Amen.}{榮福歸父,及一聖子,\\罷慰聖神,昔今永世。亞孟。}
\psAntiphonRepeat{Quem vidístis, pastóres? dícite, annuntiáte nobis, quis appáruit in terris? Natum vídimus, et choros Angelórum collaudántes Dóminum, allelúja, allelúja.}{牧童奚見?請示我等見世為疇?吾見已生者,並衆天神合讚美主。亞勒路亞,亞勒路亞。}
\psPsalmTitle{Psalmus 53}{聖詠53}
\psVerse{Deus, in nómine tuo salvum me fac: * \\et in virtúte tua júdica me.}{天主因爾名救我,\\而因義德判我。}
\psVerse{Deus, exáudi oratiónem meam: * \\áuribus pércipe verba oris mei.}{天主俯聽我禱,\\以耳明納我言。}
\psVerse{Quóniam aliéni insurrexérunt advérsum me, et fortes quæsiérunt ánimam meam: * \\et non proposuérunt Deum ante conspéctum suum.}{外人多起攻予,而猛者圖予命,\\其不揣天主在即前。}
\psVerse{Ecce enim Deus ádjuvat me: * \\et Dóminus suscéptor est ánimæ meæ.}{然天主即扶祐,\\而收予靈命。}
\psVerse{Avérte mala inimícis meis: * \\et in veritáte tua dispérde illos.}{諸仇害者望加於彼,\\因主誠實以散亡彼。}
\psVerse{Voluntárie sacrificábo tibi, * \\et confitébor nómini tuo, Dómine: quóniam bonum est:}{主予甘願獻主,而讚頌主名,\\此為美善矣。}
\psVerse{Quóniam ex omni tribulatióne eripuísti me: * \\et super inimícos meos despéxit óculus meus.}{盖主已救我於諸逆境,\\而予冷視諸仇。}
\psGloria{Glória Patri, et Fílio, * et Spirítui Sancto. \\Sicut erat in princípio, et nunc, et semper, * et in sǽcula sæculórum. Amen.}{欽頌榮福於罷德肋、及費畧、及斯彼利多三多,\\若今茲、若永遠、及無窮世。亞孟。}
\psPsalmTitle{Psalmus 118(1-16)}{聖詠118 (第一分)}
\psVerse{Beáti immaculáti in via: * \\qui ámbulant in lege Dómini.}{在路無污染而行主令,\\福哉斯人。}
\psVerse{Beáti, qui scrutántur testimónia ejus: * \\in toto corde exquírunt eum.}{蒐繹真典,以全心訪求主,\\福哉。}
\psVerse{Non enim qui operántur iniquitátem, * \\in viis ejus ambulavérunt.}{凡行惡者,\\則非行厥道。}
\psVerse{Tu mandásti * \\mandáta tua custodíri nimis.}{爾包命\\恪遵教令至矣。}
\psVerse{Útinam dirigántur viæ meæ, * \\ad custodiéndas justificatiónes tuas!}{切望引吾諸道,\\使守爾諸義者,}
\psVerse{Tunc non confúndar, * \\cum perspéxero in ómnibus mandátis tuis.}{是殆不負屈。\\熟識教令,}
\psVerse{Confitébor tibi in directióne cordis: * \\in eo quod dídici judícia justítiæ tuæ.}{因習知諸義德之判,\\將直心稱頌主。}
\psVerse{Justificatiónes tuas custódiam: * \\non me derelínquas usquequáque.}{願守主令,\\勿全棄我。}
\psVerse{In quo córrigit adolescéntior viam suam? * \\in custodiéndo sermónes tuos.}{少年改其道以何事乎?\\守主諭是也。}
\psVerse{In toto corde meo exquisívi te: * \\ne repéllas me a mandátis tuis.}{以全心訪主,\\勿驅我於爾令之外。}
\psVerse{In corde meo abscóndi elóquia tua: * \\ut non peccem tibi.}{諸令包隱於心,\\免得罪主。}
\psVerse{Benedíctus es, Dómine: * \\doce me justificatiónes tuas.}{主爾為殊福,\\望以成諸義者訓示。}
\psVerse{In lábiis meis, * \\pronuntiávi ómnia judícia oris tui.}{既以吾口\\傳告爾口諸判,}
\psVerse{In via testimoniórum tuórum delectátus sum, * \\sicut in ómnibus divítiis.}{悅樂於真典,\\如獲多財產然。}
\psVerse{In mandátis tuis exercébor: * \\et considerábo vias tuas.}{將習於諸令,\\而繹思諸道。}
\psVerse{In justificatiónibus tuis meditábor: * \\non obliviscar sermónes tuos.}{默存諸聖義之訓,\\不諼厥旨。}
\psGloria{Glória Patri, et Fílio, * et Spirítui Sancto. \\Sicut erat in princípio, et nunc, et semper, * et in sǽcula sæculórum. Amen.}{欽頌榮福於罷德肋、及費畧、及斯彼利多三多,\\若今茲、若永遠、及無窮世。亞孟。}
\psPsalmTitle{Psalmus 118(17-32)}{聖詠118 (第二分)}
\psVerse{Retríbue servo tuo, vivífica me: * \\et custódiam sermónes tuos:}{償賜爾僕得活,\\庶守主諭。}
\psVerse{Revéla óculos meos: * \\et considerábo mirabília de lege tua.}{啟吾目,\\以默存聖教奧異。}
\psVerse{Íncola ego sum in terra: * \\non abscóndas a me mandáta tua.}{予居世行旅,\\勿隱厥詔。}
\psVerse{Concupívit ánima mea desideráre justificatiónes tuas, * \\in omni témpore.}{予靈願慕主義\\時時。}
\psVerse{Increpásti supérbos: * \\maledícti qui declínant a mandátis tuis.}{主責罰傲者,\\凡不遵主令皆禍人也。}
\psVerse{Aufer a me oppróbrium, et contémptum: * \\quia testimónia tua exquisívi.}{望除於玷辱,\\盖已訪真典。}
\psVerse{Étenim sedérunt príncipes, et advérsum me loquebántur: * \\servus autem tuus exercebátur in justificatiónibus tuis.}{鉅公坐而誣我,\\爾僕乃習爾義者。}
\psVerse{Nam et testimónia tua meditátio mea est: * \\et consílium meum justificatiónes tuæ.}{真典即是予所默存,\\而予籌度全藉爾義訓。}
\psVerse{Adhǽsit paviménto ánima mea: * \\vivífica me secúndum verbum tuum.}{予靈附合於地,\\望如爾言之明告予諸途。}
\psVerse{Vias meas enuntiávi, et exaudísti me: * \\doce me justificatiónes tuas.}{乞俯聽,\\而以諸義誨之。}
\psVerse{Viam justificatiónum tuárum ínstrue me: * \\et exercébor in mirabílibus tuis.}{引於諸義之道,\\將習以爾奧異。}
\psVerse{Dormitávit ánima mea præ tǽdio: * \\confírma me in verbis tuis.}{予靈憂怯昏寐,\\乞堅定之於主訓。}
\psVerse{Viam iniquitátis ámove a me: * \\et de lege tua miserére mei.}{賜我離行惡之途,\\而因主令矜憐。}
\psVerse{Viam veritátis elégi: * \\judícia tua non sum oblítus.}{將擇真實之途,\\而不忘主諸令。}
\psVerse{Adhǽsi testimóniis tuis, Dómine: * \\noli me confúndere.}{密抱主訓,\\將不許我玷負。}
\psVerse{Viam mandatórum tuórum cucúrri, * \\cum dilatásti cor meum.}{主既寬豁予心,\\即疾趨諸令之路。}
\psGloria{Glória Patri, et Fílio, * et Spirítui Sancto. \\Sicut erat in princípio, et nunc, et semper, * et in sǽcula sæculórum. Amen.}{欽頌榮福於罷德肋、及費畧、及斯彼利多三多,\\若今茲、若永遠、及無窮世。亞孟。}
\psAntiphonRepeat{Quem vidístis, pastóres? dícite, annuntiáte nobis, quis appáruit in terris? Natum vídimus, et choros Angelórum collaudántes Dóminum, allelúja, allelúja.}{牧童奚見?請示我等見世為疇?吾見已生者,並衆天神合讚美主。亞勒路亞,亞勒路亞。}
\psRubric{Capitulum}{節目}
\psHeaderThree{1 Tim. 1:17}{弟前一17}
\psText{Regi sæculórum immortáli et invisíbili, soli Deo honor et glória in sǽcula sæculórum. Amen.}{尊敬榮福,歸於惟一天主,永世之王,永不滅,非人目得見。亞孟。}
\psVR{R}{Deo grátias.}{謝天主。}
\psRubric{Responsorium Breve}{短應}
\psVR{R}{Christe, Fili Dei vivi, * \\Miserére nobis.}{活天主子基利斯督, * \\矜憐我等。}
\psRubric{Repetitur}{重念}
\psText{Christe, Fili Dei vivi, miserére nobis.}{活天主子基利斯督,矜憐我等。}
\psVR{V}{Qui natus es de María Vírgine.}{生於童女瑪利亞。}
\psVR{R}{Miserére nobis.}{矜憐我等。}
\psVR{V}{Glória Patri, et Fílio, et Spirítui Sancto.}{欽頌榮福罷德肋、及費畧、及斯彼利多三多。}
\psVR{R}{Christe, Fili Dei vivi, miserére nobis.}{活天主子基利斯督,矜憐我等。}
\psVR{V}{Exsúrge, Christe, ádjuva nos.}{基利斯督扶祐我等。}
\psVR{R}{Et líbera nos propter nomen tuum.}{因主名救我等。}
\psRubric{Oratio}{祝文}
\psVR{V}{Dóminus vobíscum.}{主與爾偕焉。}
\psVR{R}{Et cum spíritu tuo.}{並於爾神。}
\psCollect{Dómine Deus omnípotens, qui ad princípium huius diéi nos perveníre fecísti: tua nos hódie salva virtúte; ut in hac die ad nullum declinémus peccátum, sed semper ad tuam justítiam faciéndam nostra procédant elóquia, dirigántur cogitatiónes et ópera. Per Dóminum nostrum Jesum Christum Fílium tuum: Qui tecum vivit et regnat in unitáte Spíritus Sancti Deus, per ómnia sǽcula sæculórum.}{全能天主,賜我等已至於今日,懇祈因爾德能救我等,不許岐向於罪。導吾思言行履仁義。為聖子我等主耶穌基利斯督,其偕爾偕斯彼利多三多,為一天主,世生世王。}
\psVR{R}{Amen.}{亞孟。}
\psVR{R}{Amen.}{亞孟。}
\psVR{V}{Dóminus vobíscum.}{主與爾偕焉。}
\psVR{R}{Et cum spíritu tuo.}{並於爾神。}
\psVR{V}{Benedicámus Dómino.}{讚美主。}
\psVR{R}{Deo grátias.}{謝天主。}
\psRubric{Lectio Brevis}{短讀經}
\psHeaderThree{Hebr. 1:11-12}{希一11~12}
\psText{Ipsi períbunt, tu autem permanébis; et omnes sicut vestiméntum veteráscent: et velut amíctum mutábis eos, et mutabúntur: tu autem idem ipse es, et anni tui non defícient.}{彼敗壞爾永存,諸如衣朽,又爾如衣將更之,即更矣;爾乃恒爾,年永不乏。}
\psVR{V}{Adjutórium nostrum in nómine Dómini.}{我等之扶祐因天主名。}
\psVR{R}{Qui fecit cælum et terram.}{造天地之主。}
\psVR{V}{Benedícite.}{請降福。}
\psVR{R}{Deus.}{天主。}
\psRubric{Benedictio}{降福}
\psText{Dóminus nos benedícat, et ab omni malo deféndat, et ad vitam perdúcat ætérnam. Et fidélium ánimæ per misericórdiam Dei requiéscant in pace.}{望天主降福於我等,而保護於諸凶惡,及引於常生之域。凡諸信者靈魂,賴天主仁慈息止安所。}
\psVR{R}{Amen.}{亞孟。}
\psRubric{(Pater noster, secreto.)}{(後默念天主經)}
\psThickRule
\psHeaderOneLowercase{Ad Tertiam}{三時經 (晨時經)}
\psRubric{(Pater noster, Ave María, secréto.)}{默念:天主經、聖母經。}
\psVR{V}{Deus, in adiutórium meum inténde.}{天主惟專於我扶祐。}
\psVR{R}{Dómine, ad adjuvándum me festína.}{主速格以救助我。}
\psGloria{Glória Patri, et Fílio, et Spirítui Sancto. Sicut erat in princípio, et nunc, et semper, et in sǽcula sæculórum. Amen. Allelúja.}{欽頌榮福於罷德肋、及費畧、及斯彼利多三多,\\若今茲、若永遠、及無窮世。亞孟。亞勒路亞。}
\psRubric{Hymnus}{聖歌}
\psHymnHeader{Nunc, Sancte, nobis, Spíritus}{望祈聖神}
\psHymnStanza{Nunc, Sancte, nobis, Spíritus, Unum Patri cum Fílio, \\Dignáre promptus íngeri Nostro refúsus péctori.}{望祈聖神與父子均,\\今日始辰降臨吾心。}
\psHymnStanza{Os, lingua, mens, sensus, vigor Confessiónem pérsonent. \\Flamméscat igne cáritas, Accéndat ardor próximos.}{身神諸司喜悅讚頌,\\愛火發騰熱入五内。}
\psHymnStanza{Jesu, tibi sit glória, Qui natus es de Vírgine, \\Cum Patre et almo Spíritu, In sempitérna sǽcula. Amen.}{至仁父允,子與父均,\\偕慰聖神,常王永世。亞孟。}
\psAntiphonRepeat{Génuit puérpera Regem, cui nomen ætérnum, et gáudia matris habens cum virginitátis honóre: nec primam símilem visa est, nec habére sequéntem, allelúja.}{童女產王,厥名惟永享母皇,並童貞之榮,尊推於前,引於後,無可擬者。亞勒路亞。}
\psPsalmTitle{Psalmus 118(33-48)}{聖詠118 (第三分)}
\psVerse{Legem pone mihi, Dómine, viam justificatiónum tuárum: * \\et exquíram eam semper.}{主,誨我以諸成義之道,\\將時時覓之。}
\psVerse{Da mihi intelléctum, et scrutábor legem tuam: * \\et custódiam illam in toto corde meo.}{賜我明達以研究主旨,\\而以全心守之。}
\psVerse{Deduc me in sémitam mandatórum tuórum: * \\quia ipsam vólui.}{引導我於爾諸行之徑,\\盖也望之。}
\psVerse{Inclína cor meum in testimónia tua: * \\et non in avarítiam.}{使予心向主真典,\\不許岐以貪。}
\psVerse{Avérte óculos meos ne vídeant vanitátem: * \\in via tua vivífica me.}{捩予目勿視虛偽,\\活之以爾道。}
\psVerse{Státue servo tuo elóquium tuum, * \\in timóre tuo.}{賜爾僕恭敬主旨,\\欽畏主。}
\psVerse{Ámputa oppróbrium meum quod suspicátus sum: * \\quia judícia tua jucúnda.}{斷絕余辱攸疑,\\盖主判皆仁。}
\psVerse{Ecce concupívi mandáta tua: * \\in æquitáte tua vivífica me.}{予愛慕主令,\\活我於諸義。}
\psVerse{Et véniat super me misericórdia tua, Dómine: * \\salutáre tuum secúndum elóquium tuum.}{降慈憫於我,\\即昔許救世者。}
\psVerse{Et respondébo exprobrántibus mihi verbum: * \\quia sperávi in sermónibus tuis.}{凡謗我者,將答曰:\\怙恃聖訓。}
\psVerse{Et ne áuferas de ore meo verbum veritátis usquequáque: * \\quia in judíciis tuis supersperávi.}{勿全絕予口真之言,\\盖亟望主斷。}
\psVerse{Et custódiam legem tuam semper: * \\in sǽculum et in sǽculum sǽculi.}{將永守主令。}
\psVerse{Et ambulábam in latitúdine: * \\quia mandáta tua exquisívi.}{因訪求主令,\\故豁然歡行。}
\psVerse{Et loquébar in testimóniis tuis in conspéctu regum: * \\et non confundébar.}{世王前予証真典\\而不負屈。}
\psVerse{Et meditábar in mandátis tuis, * \\quæ diléxi.}{予默存攸愛厥諸令,}
\psVerse{Et levávi manus meas ad mandáta tua, quæ diléxi: * \\et exercébar in justificatiónibus tuis.}{予舉手於攸愛者主令,\\且習以諸成義之道。}
\psGloria{Glória Patri, et Fílio, * et Spirítui Sancto. \\Sicut erat in princípio, et nunc, et semper, * et in sǽcula sæculórum. Amen.}{欽頌榮福於罷德肋、及費畧、及斯彼利多三多,\\若今茲、若永遠、及無窮世。亞孟。}
\psPsalmTitle{Psalmus 118(49-64)}{聖詠118 (第四分)}
\psVerse{Memor esto verbi tui servo tuo, * \\in quo mihi spem dedísti.}{乞憶昔攸許爾僕,\\使得所望。}
\psVerse{Hæc me consoláta est in humilitáte mea: * \\quia elóquium tuum vivificávit me.}{予居困中斯望則慰,\\盖主言已活我。}
\psVerse{Supérbi iníque agébant usquequáque: * \\a lege autem tua non declinávi.}{傲者行惡極矣,\\予則不僻犯主令。}
\psVerse{Memor fui judiciórum tuórum a sǽculo, Dómine: * \\et consolátus sum.}{主予記爾從初諸判,\\而愉快。}
\psVerse{Deféctio ténuit me, * \\pro peccatóribus derelinquéntibus legem tuam.}{因罪人棄厥教令,\\驚憂及予。}
\psVerse{Cantábiles mihi erant justificatiónes tuæ, * \\in loco peregrinatiónis meæ.}{予行旅所,\\厥諸義寵即所當歌。}
\psVerse{Memor fui nocte nóminis tui, Dómine: * \\et custodívi legem tuam.}{主予夜中憶主名,\\守厥教令。}
\psVerse{Hæc facta est mihi: * \\quia justificatiónes tuas exquisívi.}{斯守成於我內,\\盖訪求諸成義之德。}
\psVerse{Pórtio mea, Dómine, * \\dixi custodíre legem tuam.}{主予曰:\\守厥教令,即是予嗣業。}
\psVerse{Deprecátus sum fáciem tuam in toto corde meo: * \\miserére mei secúndum elóquium tuum.}{予以全心求,\\如爾言矜憐之。}
\psVerse{Cogitávi vias meas: * \\et convérti pedes meos in testimónia tua.}{繹思予諸徑,\\乃回步向爾証。}
\psVerse{Parátus sum, et non sum turbátus: * \\ut custódiam mandáta tua.}{予全備而無搖亂,\\以守諸令。}
\psVerse{Funes peccatórum circumpléxi sunt me: * \\et legem tuam non sum oblítus.}{罪索圍縛我,\\而不忘厥令。}
\psVerse{Média nocte surgébam ad confiténdum tibi, * \\super judícia justificatiónis tuæ.}{夜半起頌主,\\係主義德之諸判。}
\psVerse{Párticeps ego sum ómnium timéntium te: * \\et custodiéntium mandáta tua.}{凡敬畏主及守主令,\\予與之偕。}
\psVerse{Misericórdia tua, Dómine, plena est terra: * \\justificatiónes tuas doce me.}{主遍地以仁慈盈滿,\\以諸成義之籠訓我。}
\psGloria{Glória Patri, et Fílio, * et Spirítui Sancto. \\Sicut erat in princípio, et nunc, et semper, * et in sǽcula sæculórum. Amen.}{欽頌榮福於罷德肋、及費畧、及斯彼利多三多,\\若今茲、若永遠、及無窮世。亞孟。}
\psPsalmTitle{Psalmus 118(65-80)}{聖詠118 (第五分)}
\psVerse{Bonitátem fecísti cum servo tuo, Dómine, * \\secúndum verbum tuum.}{主,已如主言\\恩賜爾僕。}
\psVerse{Bonitátem, et disciplínam, et sciéntiam doce me: * \\quia mandátis tuis crédidi.}{誨我善及規及知識,\\盖予信奉諸令。}
\psVerse{Priúsquam humiliárer ego delíqui: * \\proptérea elóquium tuum custodívi.}{予於被責先已得罪,\\故今守主訓。}
\psVerse{Bonus es tu: * \\et in bonitáte tua doce me justificatiónes tuas.}{主乃純善,\\循爾本善,以諸義訓之。}
\psVerse{Multiplicáta est super me iníquitas superbórum: * \\ego autem in toto corde meo scrutábor mandáta tua.}{傲恣無義加於我者日長,\\予也以全心究諸令。}
\psVerse{Coagulátum est sicut lac cor eórum: * \\ego vero legem tuam meditátus sum.}{厥心凝如乳,\\予也默思主令。}
\psVerse{Bonum mihi quia humiliásti me: * \\ut discam justificatiónes tuas.}{爾責服我則利之也,\\得習諸成義之訓。}
\psVerse{Bonum mihi lex oris tui, * \\super míllia auri et argénti.}{口所出令利我哉,\\不啻千金。}
\psVerse{Manus tuæ fecérunt me, et plasmavérunt me: * \\da mihi intelléctum, et discam mandáta tua.}{製造我之窑以成我,\\乞賜明達以習諸令。}
\psVerse{Qui timent te vidébunt me, et lætabúntur: * \\quia in verba tua supersperávi.}{敬畏主者將視我而樂,\\盖予倚恃主令极矣。}
\psVerse{Cognóvi, Dómine, quia ǽquitas judícia tua: * \\et in veritáte tua humiliásti me.}{主予已騐諸判公平,\\而因其真實以責我。}
\psVerse{Fiat misericórdia tua ut consolétur me, * \\secúndum elóquium tuum servo tuo.}{如昔所許與爾僕,\\希以仁慈成於我內而慰之。}
\psVerse{Véniant mihi miseratiónes tuæ, et vivam: * \\quia lex tua meditátio mea est.}{降與仁慈庶得活,\\盖主諸令即吾所默存者。}
\psVerse{Confundántur supérbi, quia injúste iniquitátem fecérunt in me: * \\ego autem exercébor in mandátis tuis.}{望傲者負屈,厥行不義於我。\\予也習主諸令。}
\psVerse{Convertántur mihi timéntes te: * \\et qui novérunt testimónia tua.}{凡敬畏主\\及識諸証者,希拱助我。}
\psVerse{Fiat cor meum immaculátum in justificatiónibus tuis, * \\ut non confúndar.}{乞心無污於成義之訓,\\使不屈辱。}
\psGloria{Glória Patri, et Fílio, * et Spirítui Sancto. \\Sicut erat in princípio, et nunc, et semper, * et in sǽcula sæculórum. Amen.}{欽頌榮福於罷德肋、及費畧、及斯彼利多三多,\\若今茲、若永遠、及無窮世。亞孟。}
\psAntiphonRepeat{Génuit puérpera Regem, cui nomen ætérnum, et gáudia matris habens cum virginitátis honóre: nec primam símilem visa est, nec habére sequéntem, allelúja.}{童女產王,厥名惟永享母皇,並童貞之榮,尊推於前,引於後,無可擬者。亞勒路亞。}
\psRubric{Capitulum}{節目}
\psHeaderThree{Hebr. 1:1-2}{希一1~2}
\psText{Multifáriam, multisque modis olim Deus loquens pátribus in prophétis: novíssime diébus istis locútus est nobis in Fílio, quem constítuit heréde universórum, per quem fecit et sǽcula.}{昔天主屢次多方,啟先知者轉語吾祖;末時斯日,由厥子嗣萬有造寰宇者,語我等。}
\psVR{R}{Deo grátias.}{謝天主。}
\psRubric{Responsorium Breve}{短應}
\psVR{R}{Verbum caro factum est, * \\Alleluia, alleluia.}{物爾朋已降為人, * \\亞勒路亞,亞勒路亞。}
\psRubric{Repetitur}{重念}
\psText{Verbum caro factum est, Alleluia, alleluia.}{物爾朋已降為人,亞勒路亞,亞勒路亞。}
\psVR{V}{Et habitávit in nobis.}{而已居吾內。}
\psVR{R}{Alleluia, alleluia.}{亞勒路亞,亞勒路亞。}
\psVR{V}{Glória Patri, et Fílio, et Spirítui Sancto.}{欽頌榮福於罷德肋、及費畧、及斯彼利多三多。}
\psVR{R}{Verbum caro factum est, Alleluia, alleluia.}{物爾朋已降為人,亞勒路亞,亞勒路亞。}
\psVR{V}{Ipse invocábit me, allelúja.}{伊將呼予亞勒路亞。}
\psVR{R}{Pater meus es tu, allelúja.}{爾乃予父亞勒路亞。}
\psRubric{Oratio}{祝文}
\psVR{V}{Dóminus vobíscum.}{主與爾偕焉。}
\psVR{R}{Et cum spíritu tuo.}{並於爾神。}
\psCollect{Concéde, quǽsumus, omnípotens Deus: ut nos Unigéniti tui nova per carnem Natívitas líberet; quos sub peccáti jugo vetústa sérvitus tenet. Per eúmdem Dóminum nostrum Jesum Christum Fílium tuum: Qui tecum vivit et regnat in unitáte Spíritus Sancti Deus, per ómnia sǽcula sæculórum.}{懇祈全能天主,彼久在罪軛,賴爾惟一子肉身新誕,幸救脫亦。為耶穌基利斯督爾子我等主,其偕爾偕斯彼利多三多,為一天主,永生永王。}
\psVR{R}{Amen.}{亞孟。}
\psVR{V}{Dóminus vobíscum.}{主與爾偕焉。}
\psVR{R}{Et cum spíritu tuo.}{並於爾神。}
\psVR{V}{Benedicámus Dómino.}{讚美主。}
\psVR{R}{Deo grátias.}{謝天主。}
\psVR{V}{Fidélium ánimæ per misericórdiam Dei requiéscant in pace.}{凡諸信者靈魂,賴天主仁慈息止安所。}
\psVR{R}{Amen.}{亞孟。}
\psRubric{(Pater noster, secreto.)}{(後默念天主經)}
\psThickRule
\psHeaderOneLowercase{Ad Sextam}{六時經 (午時經)}
\psRubric{(Pater noster, Ave María, secréto.)}{默念:天主經、聖母經。}
\psVR{V}{Deus, in adiutórium meum inténde.}{天主惟專於我扶祐。}
\psVR{R}{Dómine, ad adjuvándum me festína.}{主速格以救助我。}
\psGloria{Glória Patri, et Fílio, et Spirítui Sancto. Sicut erat in princípio, et nunc, et semper, et in sǽcula sæculórum. Amen. Allelúja.}{欽頌榮福於罷德肋、及費畧、及斯彼利多三多,\\若今茲、若永遠、及無窮世。亞孟。亞勒路亞。}
\psRubric{Hymnus}{聖歌}
\psHymnHeader{Rector potens, verax Deus}{治宰真主}
\psHymnStanza{Rector potens, verax Deus, Qui témperas rerum vices, \\Splendóre mane illúminas, Et ígnibus merídiem:}{治宰真主,時日錯行。\\晨以光照,午以火煖。}
\psHymnStanza{Exstíngue flammas lítium, Aufer calórem nóxium, \\Confer salútem córporum, Verámque pacem córdium.}{滅我怒火,除消毒熱。\\賜身康寧,及心平和。}
\psHymnStanza{Jesu, tibi sit glória, Qui natus es de Vírgine, \\Cum Patre et almo Spíritu, In sempitérna sǽcula. Amen.}{至仁父允,子與父均,\\偕慰聖神,常王永世。亞孟。}
\psAntiphonRepeat{Angelus ad pastóres ait: Annúntio vobis gáudium magnum: quia natus est vobis hódie Salvátor mundi, allelúja.}{天神向牧童曰:予來報爾莫大喜音,救世主為爾適誕。亞勒路亞。}
\psPsalmTitle{Psalmus 118(81-96)}{聖詠118 (第六分)}
\psVerse{Defécit in salutáre tuum ánima mea: * \\et in verbum tuum supersperávi.}{予靈既竭於救世者,\\倚恃主令極矣。}
\psVerse{Defecérunt óculi mei in elóquium tuum, * \\dicéntes: Quando consoláberis me?}{予目盼主約乏矣,\\曰:於何時安慰哉?}
\psVerse{Quia factus sum sicut uter in pruína: * \\justificatiónes tuas non sum oblítus.}{予如皮囊漬於霜,\\莫忘諸成義之訓。}
\psVerse{Quot sunt dies servi tui? * \\quando fácies de persequéntibus me judícium?}{爾僕之日幾何乎?\\何時主判諸害我者?}
\psVerse{Narravérunt mihi iníqui fabulatiónes: * \\sed non ut lex tua.}{惡人語我虛事,\\匪從主令。}
\psVerse{Ómnia mandáta tua véritas: * \\iníque persecúti sunt me, ádjuva me.}{主諸令即真實,\\無義者踵而圖害主,救我。}
\psVerse{Paulo minus consummavérunt me in terra: * \\ego autem non derelíqui mandáta tua.}{殆將滅我於世,\\予不廢諸令。}
\psVerse{Secúndum misericórdiam tuam vivífica me: * \\et custódiam testimónia oris tui.}{如爾仁慈活我,\\乃守爾口真証。}
\psVerse{In ætérnum, Dómine, * \\verbum tuum pérmanet in cælo.}{主爾旨永行於天,}
\psVerse{In generatiónem et generatiónem véritas tua: * \\fundásti terram, et pérmanet.}{爾真實於繼世,\\置基地而存。}
\psVerse{Ordinatióne tua persevérat dies: * \\quóniam ómnia sérviunt tibi.}{因主制令日行有常,\\萬有奉事主。}
\psVerse{Nisi quod lex tua meditátio mea est: * \\tunc forte periíssem in humilitáte mea.}{若主令余非默存,\\困中將亡散。}
\psVerse{In ætérnum non obliviscar justificatiónes tuas: * \\quia in ipsis vivificásti me.}{予永不忘成義之訓,\\乃爾以厥活我。}
\psVerse{Tuus sum ego, salvum me fac: * \\quóniam justificatiónes tuas exquisívi.}{予是主物,造救我,\\盖色訪諸義訓。}
\psVerse{Me exspectavérunt peccatóres ut pérderent me: * \\testimónia tua intelléxi.}{罪人伺我以亡,\\予則明主真証。}
\psVerse{Omnis consummatiónis vidi finem: * \\latum mandátum tuum nimis.}{凡屬壞之物,予已見其終,\\爾令寬延至矣。}
\psGloria{Glória Patri, et Fílio, * et Spirítui Sancto. \\Sicut erat in princípio, et nunc, et semper, * et in sǽcula sæculórum. Amen.}{欽頌榮福於罷德肋、及費畧、及斯彼利多三多,\\若今茲、若永遠、及無窮世。亞孟。}
\psPsalmTitle{Psalmus 118(97-112)}{聖詠118 (第七分)}
\psVerse{Quómodo diléxi legem tuam, Dómine? * \\tota die meditátio mea est.}{主予曷愛主令哉!\\終日吾所默存。}
\psVerse{Super inimícos meos prudéntem me fecísti mandáto tuo: * \\quia in ætérnum mihi est.}{奉主令,賜我明智,越諸仇能,\\矢永安。}
\psVerse{Super omnes docéntes me intelléxi: * \\quia testimónia tua meditátio mea est.}{予智識越凡講誨者,\\主真証即予所默存。}
\psVerse{Super senes intelléxi: * \\quia mandáta tua quæsívi.}{予明邁耆年者,\\盖訪循主諸訓。}
\psVerse{Ab omni via mala prohíbui pedes meos: * \\ut custódiam verba tua.}{禁錯趾於諸惡徑,\\以守主言。}
\psVerse{A judíciis tuis non declinávi: * \\quia tu legem posuísti mihi.}{莫岐趨於訓,\\既箴規予。}
\psVerse{Quam dúlcia fáucibus meis elóquia tua, * \\super mel ori meo!}{爾語怡予喉至哉,\\予口飴於蜜哉。}
\psVerse{A mandátis tuis intelléxi: * \\proptérea odívi omnem viam iniquitátis.}{予明習主諸訓,\\故憎行諸惡徑。}
\psVerse{Lucérna pédibus meis verbum tuum, * \\et lumen sémitis meis.}{主訓為予足之燈,\\予步之光。}
\psVerse{Jurávi, et státui * \\custodíre judícia justítiæ tuæ.}{矢堅守\\主諸義之訓。}
\psVerse{Humiliátus sum usquequáque, Dómine: * \\vivífica me secúndum verbum tuum.}{主予負屈至矣,\\乞因主言活之。}
\psVerse{Voluntária oris mei beneplácita fac, Dómine: * \\et judícia tua doce me.}{主納予口所願,\\而以諸判訓之。}
\psVerse{Ánima mea in mánibus meis semper: * \\et legem tuam non sum oblítus.}{予靈時在予手,\\然不忘厥令。}
\psVerse{Posuérunt peccatóres láqueum mihi: * \\et de mandátis tuis non errávi.}{造孽者設陷阱於我,\\我也不迷於厥令。}
\psVerse{Hereditáte acquisívi testimónia tua in ætérnum: * \\quia exsultátio cordis mei sunt.}{主証已為予永嗣業,\\盖為予心之愉。}
\psVerse{Inclinávi cor meum ad faciéndas justificatiónes tuas in ætérnum, * \\propter retributiónem.}{予引向予心,以永守主令,\\為報之慰。}
\psGloria{Glória Patri, et Fílio, * et Spirítui Sancto. \\Sicut erat in princípio, et nunc, et semper, * et in sǽcula sæculórum. Amen.}{欽頌榮福於罷德肋、及費畧、及斯彼利多三多,\\若今茲、若永遠、及無窮世。亞孟。}
\psPsalmTitle{Psalmus 118(113-128)}{聖詠118 (第八分)}
\psVerse{Iníquos ódio hábui: * \\et legem tuam diléxi.}{予既憎無義人,\\而愛主令。}
\psVerse{Adjútor et suscéptor meus es tu: * \\et in verbum tuum supersperávi.}{主為吾保存畏避者,\\予望主言切矣。}
\psVerse{Declináte a me, malígni: * \\et scrutábor mandáta Dei mei.}{逆黨即避去,\\我將考詳予天主令。}
\psVerse{Súscipe me secúndum elóquium tuum, et vivam: * \\et non confúndas me ab exspectatióne mea.}{因主訓接而懷之,乃活。\\無失望救我,}
\psVerse{Ádjuva me, et salvus ero: * \\et meditábor in justificatiónibus tuis semper.}{我即安時,\\將默存諸義。}
\psVerse{Sprevísti omnes discedéntes a judíciis tuis: * \\quia injústa cogitátio eórum.}{凡違諸義者,爾賤棄之,\\盖常蓄慮而不義。}
\psVerse{Prævaricántes reputávi omnes peccatóres terræ: * \\ídeo diléxi testimónia tua.}{世之負恩者,予識為罪人,\\故予已愛守諸証。}
\psVerse{Confíge timóre tuo carnes meas: * \\a judíciis enim tuis tímui.}{望以主威刺身,\\予懔懔諸判。}
\psVerse{Feci judícium et justítiam: * \\non tradas me calumniántibus me.}{既自施義判,\\勿付於諸誣誑者。}
\psVerse{Súscipe servum tuum in bonum: * \\non calumniéntur me supérbi.}{以仁慈接懷爾僕,\\禁傲惡者誣污予目。}
\psVerse{Óculi mei defecérunt in salutáre tuum: * \\et in elóquium justítiæ tuæ.}{竭於救世者\\及爾義約矣。}
\psVerse{Fac cum servo tuo secúndum misericórdiam tuam: * \\et justificatiónes tuas doce me.}{循爾慈加爾僕,\\而以成諸義者訓之。}
\psVerse{Servus tuus sum ego: * \\da mihi intelléctum, ut sciam testimónia tua.}{予即爾僕也,\\賜明達以証諸真典。}
\psVerse{Tempus faciéndi, Dómine: * \\dissipavérunt legem tuam.}{主行為延遲,\\多衆散亂主令。}
\psVerse{Ídeo diléxi mandáta tua, * \\super aurum et topázion.}{予愛主令,\\踰於黄金及多巴爵寶石。}
\psVerse{Proptérea ad ómnia mandáta tua dirigébar: * \\omnem viam iníquam ódio hábui.}{故予所向惟諸教令,\\凡諸惡行予疾惡者。}
\psGloria{Glória Patri, et Fílio, * et Spirítui Sancto. \\Sicut erat in princípio, et nunc, et semper, * et in sǽcula sæculórum. Amen.}{欽頌榮福於罷德肋、及費畧、及斯彼利多三多,\\若今茲、若永遠、及無窮世。亞孟。}
\psAntiphonRepeat{Angelus ad pastóres ait: Annúntio vobis gáudium magnum: quia natus est vobis hódie Salvátor mundi, allelúja.}{天神向牧童曰:予來報爾莫大喜音,救世主為爾適誕。亞勒路亞。}
\psRubric{Capitulum}{節目}
\psHeaderThree{Hebr. 1:10}{希一10}
\psText{Et: Tu in princípio, Dómine, terram fundásti: et ópera mánuum tuárum sunt cæli.}{爾厥始創基大地,諸天即為爾手之工。}
\psVR{R}{Deo grátias.}{謝天主。}
\psRubric{Responsorium Breve}{短應}
\psVR{R}{Notum fecit Dóminus, * \\Alleluia, alleluia.}{主已顯, * \\亞勒路亞,亞勒路亞。}
\psRubric{Repetitur}{重念}
\psText{Notum fecit Dóminus, Alleluia, alleluia.}{主已顯,亞勒路亞,亞勒路亞。}
\psVR{V}{Salutáre suum.}{厥救世者。}
\psVR{R}{Alleluia, alleluia.}{亞勒路亞,亞勒路亞。}
\psVR{V}{Glória Patri, et Fílio, et Spirítui Sancto.}{欽頌榮福於罷德肋、及費畧、及斯彼利多三多。}
\psVR{R}{Notum fecit Dóminus, Alleluia, alleluia.}{主已顯,亞勒路亞,亞勒路亞。}
\psVR{V}{Vidérunt omnes términi terræ, allelúja.}{遍地末境皆已見,亞勒路亞。}
\psVR{R}{Salutáre Dei nostri, allelúja.}{吾天主救世者,亞勒路亞。}
\psRubric{Oratio}{祝文}
\psVR{V}{Dóminus vobíscum.}{主與爾偕焉。}
\psVR{R}{Et cum spíritu tuo.}{並於爾神。}
\psCollect{Concéde, quǽsumus, omnípotens Deus: ut nos Unigéniti tui nova per carnem Natívitas líberet; quos sub peccáti jugo vetústa sérvitus tenet. Per eúmdem Dóminum nostrum Jesum Christum Fílium tuum: Qui tecum vivit et regnat in unitáte Spíritus Sancti Deus, per ómnia sǽcula sæculórum.}{懇祈全能天主,彼久在罪軛,賴爾惟一子肉身新誕,幸救脫亦。為耶穌基利斯督爾子我等主,其偕爾偕斯彼利多三多,為一天主,永生永王。}
\psVR{R}{Amen.}{亞孟。}
\psVR{V}{Dóminus vobíscum.}{主與爾偕焉。}
\psVR{R}{Et cum spíritu tuo.}{並於爾神。}
\psVR{V}{Benedicámus Dómino.}{讚美主。}
\psVR{R}{Deo grátias.}{謝天主。}
\psVR{V}{Fidélium ánimæ per misericórdiam Dei requiéscant in pace.}{凡諸信者靈魂,賴天主仁慈息止安所。}
\psVR{R}{Amen.}{亞孟。}
\psRubric{(Pater noster, secreto.)}{(後默念天主經)}
\psThickRule
\psHeaderOneLowercase{Ad Nonam}{九時經 (申初經)}
\psRubric{(Pater noster, Ave María, secréto.)}{默念:天主經、聖母經。}
\psVR{V}{Deus, in adiutórium meum inténde.}{天主惟專於我扶祐。}
\psVR{R}{Dómine, ad adjuvándum me festína.}{主速格以救助我。}
\psGloria{Glória Patri, et Fílio, et Spirítui Sancto. Sicut erat in princípio, et nunc, et semper, et in sǽcula sæculórum. Amen. Allelúja.}{欽頌榮福於罷德肋、及費畧、及斯彼利多三多,\\若今茲、若永遠、及無窮世。亞孟。亞勒路亞。}
\psRubric{Hymnus}{聖歌}
\psHymnHeader{Rerum Deus tenax vigor}{造物真主}
\psHymnStanza{Rerum Deus tenax vigor, Immótus in te pérmanens, \\Lucis diúrnæ témpora Succéssibus detérminans:}{造物真主,萬用靈運,\\本恒凝一,曦暉迭旋。}
\psHymnStanza{Largíre lumen véspere, Quo vita nusquam décidat, \\Sed prǽmium mortis sacræ Perénnis instet glória.}{盼懇罷錫,不沒神光,\\得以永榮,酬其善死。}
\psHymnStanza{Præsta, Pater piíssime, Patríque compar Unice, \\Cum Spíritu Paráclito Regnans per omne sǽculum. Amen.}{至仁父允,子與父均,\\偕慰聖神,常王永世。亞孟。}
\psAntiphonRepeat{Párvulus fílius hódie natus est nobis: et vocábitur Deus, Fortis, allelúja.}{今日嬰孩為吾生誕,將稱勇天主。亞勒路亞。}
\psPsalmTitle{Psalmus 118(129-144)}{聖詠118 (第九分)}
\psVerse{Mirabília testimónia tua: * \\ídeo scrutáta est ea ánima mea.}{主諸証奇哉,\\以吾靈訪究之。}
\psVerse{Declarátio sermónum tuórum illúminat: * \\et intelléctum dat párvulis.}{訓典之詮,且施光照,\\而開明悟於幼穉。}
\psVerse{Os meum apérui, et attráxi spíritum: * \\quia mandáta tua desiderábam.}{予啟口吸氣,\\盖覬主諸令。}
\psVerse{Áspice in me, et miserére mei, * \\secúndum judícium diligéntium nomen tuum.}{垂視而矜憐之,\\如愛爾名者之意。}
\psVerse{Gressus meos dírige secúndum elóquium tuum: * \\et non dominétur mei omnis injustítia.}{因主訓指示予步,\\禁諸惡德宰制我。}
\psVerse{Rédime me a calúmniis hóminum: * \\ut custódiam mandáta tua.}{贖救於誣我者,\\使守諸令。}
\psVerse{Fáciem tuam illúmina super servum tuum: * \\et doce me justificatiónes tuas.}{聖容光照爾僕,\\而以諸義德訓之。}
\psVerse{Exitus aquárum deduxérunt óculi mei: * \\quia non custodiérunt legem tuam.}{予目湧淚,\\因視違令。}
\psVerse{Justus es, Dómine: * \\et rectum judícium tuum.}{主乃至義,\\厥判並至義命。}
\psVerse{Mandásti justítiam testimónia tua: * \\et veritátem tuam nimis.}{受厥令為至義\\至真。}
\psVerse{Tabéscere me fecit zelus meus: * \\quia oblíti sunt verba tua inimíci mei.}{予愛憤乃使予癯瘠,\\盖吾仇忘主諭。}
\psVerse{Ignítum elóquium tuum veheménter: * \\et servus tuus diléxit illud.}{主諭如火熾,\\爾僕則愛之。}
\psVerse{Adolescéntulus sum ego et contémptus: * \\justificatiónes tuas non sum oblítus.}{爾僕幼穉人,乃微賤之,\\而不忘主諸義。}
\psVerse{Justítia tua, justítia in ætérnum: * \\et lex tua véritas.}{主之義永義也,\\主命皆誠也。}
\psVerse{Tribulátio, et angústia invenérunt me: * \\mandáta tua meditátio mea est.}{憂患追於我躬,\\惟默存主命。}
\psVerse{Ǽquitas testimónia tua in ætérnum: * \\intelléctum da mihi, et vivam.}{真典公平永世,\\懇賜寵悟,予即得生。}
\psGloria{Glória Patri, et Fílio, * et Spirítui Sancto. \\Sicut erat in princípio, et nunc, et semper, * et in sǽcula sæculórum. Amen.}{欽頌榮福於罷德肋、及費畧、及斯彼利多三多,\\若今茲、若永遠、及無窮世。亞孟。}
\psPsalmTitle{Psalmus 118(145-160)}{聖詠118 (第十分)}
\psVerse{Clamávi in toto corde meo, exáudi me, Dómine: * \\justificatiónes tuas requíram.}{以全心呼號主,爾俯聽之,\\將訪主諸義。}
\psVerse{Clamávi ad te, salvum me fac: * \\ut custódiam mandáta tua.}{呼號主救,\\以守諸令。}
\psVerse{Prævéni in maturitáte, et clamávi: * \\quia in verba tua supersperávi.}{先治備而呼號,\\盖切恃主諭。}
\psVerse{Prævenérunt óculi mei ad te dilúculo: * \\ut meditárer elóquia tua.}{予清晨盼迎主,\\以默存主諭。}
\psVerse{Vocem meam audi secúndum misericórdiam tuam, Dómine: * \\et secúndum judícium tuum vivífica me.}{主循爾慈俯聽呼號,\\循其定判寵活我。}
\psVerse{Appropinquavérunt persequéntes me iniquitáti: * \\a lege autem tua longe facti sunt.}{諸害我者邇於不義,\\是遠於主令。}
\psVerse{Prope es tu, Dómine: * \\et omnes viæ tuæ véritas.}{主實邇,\\諸徑皆真。}
\psVerse{Inítio cognóvi de testimóniis tuis: * \\quia in ætérnum fundásti ea.}{首初予識真典,\\乃從無始置定。}
\psVerse{Vide humilitátem meam, et éripe me: * \\quia legem tuam non sum oblítus.}{憐視予屈辱而提救之,\\盖不忘主令。}
\psVerse{Júdica judícium meum, et rédime me: * \\propter elóquium tuum vivífica me.}{判是非而救之,\\因主言活之。}
\psVerse{Longe a peccatóribus salus: * \\quia justificatiónes tuas non exquisiérunt.}{惡黨遠於真福,\\莫訪諸成義者。}
\psVerse{Misericórdiæ tuæ multæ, Dómine: * \\secúndum judícium tuum vivífica me.}{主仁慈衆矣,\\因其定判活我。}
\psVerse{Multi qui persequúntur me, et tríbulant me: * \\a testimóniis tuis non declinávi.}{多衆害苦我,\\而予莫岐於諸真証。}
\psVerse{Vidi prævaricántes, et tabescébam: * \\quia elóquia tua non custodiérunt.}{予觀犯令者則癯瘠,\\因不守真証。}
\psVerse{Vide quóniam mandáta tua diléxi, Dómine: * \\in misericórdia tua vivífica me.}{主鑒予愛厥令,\\懇主仁慈活之。}
\psVerse{Princípium verbórum tuórum véritas: * \\in ætérnum ómnia judícia justítiæ tuæ.}{主諭始即誠實,\\諸義判永世。}
\psGloria{Glória Patri, et Fílio, * et Spirítui Sancto. \\Sicut erat in princípio, et nunc, et semper, * et in sǽcula sæculórum. Amen.}{欽頌榮福於罷德肋、及費畧、及斯彼利多三多,\\若今茲、若永遠、及無窮世。亞孟。}
\psPsalmTitle{Psalmus 118(161-176)}{聖詠118 (第十一分)}
\psVerse{Príncipes persecúti sunt me gratis: * \\et a verbis tuis formidávit cor meum.}{鉅公無故肆虐,\\然予心悚惕主言。}
\psVerse{Lætábor ego super elóquia tua: * \\sicut qui invénit spólia multa.}{予歡慰主諭,\\如勝軍掠敵。}
\psVerse{Iniquitátem ódio hábui, et abominátus sum: * \\legem autem tuam diléxi.}{予憎無義且屏斥之,\\而愛主令。}
\psVerse{Sépties in die laudem dixi tibi, * \\super judícia justítiæ tuæ.}{日中七次讚美主,\\為主義德之定案。}
\psVerse{Pax multa diligéntibus legem tuam: * \\et non est illis scándalum.}{愛主令者多平和,\\彼無何誘阻。}
\psVerse{Exspectábam salutáre tuum, Dómine: * \\et mandáta tua diléxi.}{主予跂望救世者之恩佑,\\且愛厥令。}
\psVerse{Custodívit ánima mea testimónia tua: * \\et diléxit ea veheménter.}{予靈守真証\\而愛之甚。}
\psVerse{Servávi mandáta tua, et testimónia tua: * \\quia omnes viæ meæ in conspéctu tuo.}{予守爾令及爾証,\\盖予諸徑在主臺前。}
\psVerse{Appropínquet deprecátio mea in conspéctu tuo, Dómine: * \\juxta elóquium tuum da mihi intelléctum.}{懇主賜予祈禱近主前,\\循主諭賜之明達。}
\psVerse{Intret postulátio mea in conspéctu tuo: * \\secúndum elóquium tuum éripe me.}{希予祈禱進主臺前,\\循主諭救之。}
\psVerse{Eructábunt lábia mea hymnum, * \\cum docúeris me justificatiónes tuas.}{以諸義示訓,\\予唇將噯聖詠。}
\psVerse{Pronuntiábit lingua mea elóquium tuum: * \\quia ómnia mandáta tua ǽquitas.}{予舌將宣聖諭,\\盖主令皆公平。}
\psVerse{Fiat manus tua ut salvet me: * \\quóniam mandáta tua elégi.}{望舉手救之,\\予已選取諸令。}
\psVerse{Concupívi salutáre tuum, Dómine: * \\et lex tua meditátio mea est.}{主予覬覦爾救世者,\\主令即是予默存。}
\psVerse{Vivet ánima mea, et laudábit te: * \\et judícia tua adjuvábunt me.}{望予靈得活讚頌主,\\主諸判將扶持我。}
\psVerse{Errávi, sicut ovis, quæ périit: * \\quǽre servum tuum, quia mandáta tua non sum oblítus.}{予錯岐如羊已亡,\\懇主訪爾僕,予莫忘諸令。}
\psGloria{Glória Patri, et Fílio, * et Spirítui Sancto. \\Sicut erat in princípio, et nunc, et semper, * et in sǽcula sæculórum. Amen.}{欽頌榮福於罷德肋、及費畧、及斯彼利多三多,\\若今茲、若永遠、及無窮世。亞孟。}
\psAntiphonRepeat{Párvulus fílius hódie natus est nobis: et vocábitur Deus, Fortis, allelúja.}{今日嬰孩為吾生誕,將稱勇天主。亞勒路亞。}
\psRubric{Capitulum}{節目}
\psHeaderThree{Hebr. 1:11-12}{希一11,12}
\psText{Ipsi períbunt, tu autem permanébis; et omnes sicut vestiméntum veteráscent: et velut amíctum mutábis eos, et mutabúntur: tu autem idem ipse es, et anni tui non defícient.}{彼敗壞爾永存,諸如衣朽,又爾如衣將更之,即更矣;爾乃恒爾,年永不乏。}
\psVR{R}{Deo grátias.}{謝天主。}
\psRubric{Responsorium Breve}{短應}
\psVR{R}{Vidérunt omnes términi terræ, * \\Alleluia, alleluia.}{遍地末境已見, * \\亞勒路亞,亞勒路亞。}
\psRubric{Repetitur}{重念}
\psText{Vidérunt omnes términi terræ, Alleluia, alleluia.}{遍地末境已見,亞勒路亞,亞勒路亞。}
\psVR{V}{Salutáre Dei nostri.}{吾天主救世者。}
\psVR{R}{Alleluia, alleluia.}{亞勒路亞,亞勒路亞。}
\psVR{V}{Glória Patri, et Fílio, et Spirítui Sancto.}{欽頌榮福於罷德肋、及費畧、及斯彼利多三多。}
\psVR{R}{Vidérunt omnes términi terræ, Alleluia, alleluia.}{遍地末境已見,亞勒路亞,亞勒路亞。}
\psVR{V}{Verbum caro factum est, allelúja.}{物爾朋已降為人,亞勒路亞。}
\psVR{R}{Et habitávit in nobis, allelúja.}{而已居吾內,亞勒路亞。}
\psRubric{Oratio}{祝文}
\psVR{V}{Dóminus vobíscum.}{主與爾偕焉。}
\psVR{R}{Et cum spíritu tuo.}{並於爾神。}
\psCollect{Concéde, quǽsumus, omnípotens Deus: ut nos Unigéniti tui nova per carnem Natívitas líberet; quos sub peccáti jugo vetústa sérvitus tenet. Per eúmdem Dóminum nostrum Jesum Christum Fílium tuum: Qui tecum vivit et regnat in unitáte Spíritus Sancti Deus, per ómnia sǽcula sæculórum.}{懇祈全能天主,彼久在罪軛,賴爾惟一子肉身新誕,幸救脫亦。為耶穌基利斯督爾子我等主,其偕爾偕斯彼利多三多,為一天主,永生永王。}
\psVR{R}{Amen.}{亞孟。}
\psVR{V}{Dóminus vobíscum.}{主與爾偕焉。}
\psVR{R}{Et cum spíritu tuo.}{並於爾神。}
\psVR{V}{Benedicámus Dómino.}{讚美主。}
\psVR{R}{Deo grátias.}{謝天主。}
\psVR{V}{Fidélium ánimæ per misericórdiam Dei requiéscant in pace.}{凡諸信者靈魂,賴天主仁慈息止安所。}
\psVR{R}{Amen.}{亞孟。}
\psRubric{(Pater noster, secreto.)}{(後默念天主經)}
\psThickRule
\psHeaderOneLowercase{Ad II Vesperas}{申正經 (係本日)}
\psRubric{(Dícitur Pater noster et Ave María, secréto.)}{默念:天主經、聖母經。\\ (畢,明聲念:)}
\psVR{V}{Deus, in adiutórium meum inténde.}{天主惟專於我扶祐。}
\psVR{R}{Dómine, ad adjuvándum me festína.}{主速格以救助我。}
\psGloria{Glória Patri, et Fílio, et Spirítui Sancto. Sicut erat in princípio, et nunc, et semper, et in sǽcula sæculórum. Amen. Allelúja.}{欽頌榮福於罷德肋、及費畧、及斯彼利多三多,\\若今茲、若永遠、及無窮世。亞孟。亞勒路亞。}
\psAntiphonRepeat{Tecum princípium in die virtútis tuæ in splendóribus sanctórum: ex útero ante lucíferum génui te.}{爾德能之日原始,偕爾諸聖之光中;啟明之先,由胎予已生爾。}
\psPsalmTitle{Psalmus 109}{聖詠109}
\psVerse{Dixit Dóminus Dómino meo: * Sede a dextris meis:}{主語吾主曰:爾坐予右,\\俟予伏爾仇於爾足下。}
\psVerse{Donec ponam inimícos tuos, * scabéllum pedum tuórum.}{主自西婉將發爾德能之鋌,\\爾將王爾仇之中。}
\psVerse{Virgam virtútis tuæ emíttet Dóminus ex Sion: * domináre in médio inimicórum tuórum.}{爾德能之日,原始偕爾諸聖之光中;\\啟明之先,予已胚胎生爾。}
\psVerse{Tecum princípium in die virtútis tuæ in splendóribus sanctórum: * ex útero ante lucíferum génui te.}{主矢而不悔:\\爾將永享撒責爾鐸德爵位,從默其塞德例。}
\psVerse{Jurávit Dóminus, et non pœnitébit eum: * Tu es sacérdos in ætérnum secúndum órdinem Melchísedech.}{主於爾右,\\厥怒之日,已勦衆師。}
\psVerse{Dóminus a dextris tuis, * confrégit in die iræ suæ reges.}{將審判衆民,補諸殘缺,\\而於普地摧諸渠魁。}
\psVerse{Judicábit in natiónibus, implébit ruínas: * conquassábit cápita in terra multórum.}{途中將飲洶流之水,\\乃得昂首。}
\psVerse{De torrénte in via bibet: * proptérea exaltábit caput.}{途中將飲洶流之水,\\乃得昂首。}
\psGloria{Glória Patri, et Fílio, * et Spirítui Sancto. \\Sicut erat in princípio, et nunc, et semper, * et in sǽcula sæculórum. Amen.}{欽頌榮福於罷德肋、及費畧、及斯彼利多三多,\\若今茲、若永遠、及無窮世。亞孟。}
\psAntiphonRepeat{Tecum princípium in die virtútis tuæ in splendóribus sanctórum: ex útero ante lucíferum génui te.}{爾德能之日原始,偕爾諸聖之光中;啟明之先,由胎予已生爾。}
\psAntiphonRepeat{Redemptiónem misit Dóminus pópulo suo: mandávit in ætérnum testaméntum suum.}{主遣救贖於厥民,令厥詔於無窮世。}
\psPsalmTitle{Psalmus 110}{聖詠110}
\psVerse{Confitébor tibi, Dómine, in toto corde meo: * in consílio justórum, et congregatióne.}{主,予以全心將讚美主,\\於義人會議之處。}
\psVerse{Magna ópera Dómini: * exquisíta in omnes voluntátes ejus.}{主工弘矣,\\盡合厥志。}
\psVerse{Conféssio et magnificéntia opus ejus: * et justítia ejus manet in sǽculum sǽculi.}{備全稱頌即厥工也,博施亦厥工也,\\而義永存。}
\psVerse{Memóriam fecit mirabílium suórum, miséricors et miserátor Dóminus: * escam dedit timéntibus se.}{仁慈主已憶其昔行靈異,\\賜食於畏主者。}
\psVerse{Memor erit in sǽculum testaménti sui: * virtútem óperum suórum annuntiábit pópulo suo:}{主永不忘厥詔,\\將示厥工之德能於其民,}
\psVerse{Ut det illis hereditátem géntium: * ópera mánuum ejus véritas, et judícium.}{以與伊等異教之嗣業。\\主之掌握,惟直惟判;}
\psVerse{Fidélia ómnia mandáta ejus: confirmáta in sǽculum sǽculi, * facta in veritáte et æquitáte.}{教令信實,永遠世騐;\\証皆真皆公。}
\psVerse{Redemptiónem misit pópulo suo: * mandávit in ætérnum testaméntum suum.}{遣救贖於衆民,\\頒詔於無窮世。}
\psVerse{Sanctum, et terríbile nomen ejus: * inítium sapiéntiæ timor Dómini.}{厥名聖哉赫哉。\\欽畏即明哲之原,}
\psVerse{Intelléctus bonus ómnibus faciéntibus eum: * laudátio ejus manet in sǽculum sǽculi.}{循其明哲而行,乃大益。\\伊等稱頌,尊於永世。}
\psGloria{Glória Patri, et Fílio, * et Spirítui Sancto. \\Sicut erat in princípio, et nunc, et semper, * et in sǽcula sæculórum. Amen.}{欽頌榮福於罷德肋、及費畧、及斯彼利多三多,\\若今茲、若永遠、及無窮世。亞孟。}
\psAntiphonRepeat{Redemptiónem misit Dóminus pópulo suo: mandávit in ætérnum testaméntum suum.}{主遣救贖於厥民,令厥詔於無窮世。}
\psAntiphonRepeat{Exórtum est in ténebris lumen rectis: miséricors, et miserátor, et justus Dóminus.}{真心者居暗已受光照,維惻維義天主。}
\psPsalmTitle{Psalmus 111}{聖詠111}
\psVerse{Beátus vir, qui timet Dóminum: * in mandátis ejus volet nimis.}{畏天主真福人哉,\\歡忭而奉詔令。}
\psVerse{Potens in terra erit semen ejus: * generátio rectórum benedicétur.}{嗣將強於世,\\義人後代將蒙殊福。}
\psVerse{Glória, et divítiæ in domo ejus: * et justítia ejus manet in sǽculum sǽculi.}{居世且榮且富,\\而厥義永存。}
\psVerse{Exórtum est in ténebris lumen rectis: * miséricors, et miserátor, et justus.}{質直者居暗已受光照,\\且惻且義。}
\psVerse{Jucúndus homo qui miserétur et cómmodat, dispónet sermónes suos in judício: * quia in ætérnum non commovébitur.}{天主哀矜,濟以財,普樂斯人哉。審判將慎厥語,\\盖永不易動。}
\psVerse{In memória ætérna erit justus: * ab auditióne mala non timébit.}{義人永憶於無窮世,\\將不懼惡聞。}
\psVerse{Parátum cor ejus speráre in Dómino, confirmátum est cor ejus: * non commovébitur donec despíciat inimícos suos.}{厥心預備既定,以望天主不易動,\\自懾厥仇。}
\psVerse{Dispérsit, dedit paupéribus: justítia ejus manet in sǽculum sǽculi, * cornu ejus exaltábitur in glória.}{伊散於貧者,義且永世,\\義德上徹於榮光。}
\psVerse{Peccátor vidébit, et irascétur, déntibus suis fremet et tabéscet: * desidérium peccatórum períbit.}{罪人觀即瞋怒,且切齒而狺憤,鬱而摧裂;\\罪人之冀望將亡。}
\psGloria{Glória Patri, et Fílio, * et Spirítui Sancto. \\Sicut erat in princípio, et nunc, et semper, * et in sǽcula sæculórum. Amen.}{欽頌榮福於罷德肋、及費畧、及斯彼利多三多,\\若今茲、若永遠、及無窮世。亞孟。}
\psAntiphonRepeat{Exórtum est in ténebris lumen rectis: miséricors, et miserátor, et justus Dóminus.}{真心者居暗已受光照,維惻維義天主。}
\psAntiphonRepeat{Apud Dóminum misericórdia: et copiósa apud eum redémptio.}{慈惻本在主,而救贖之功遍滿矣。}
\psPsalmTitle{Psalmus 129}{聖詠129}
\psVerse{De profúndis clamávi ad te, Dómine: * Dómine, exáudi vocem meam:}{主予自幽幽已籲號爾。\\主俯聽我禱,}
\psVerse{Fiant aures tuæ intendéntes, * in vocem deprecatiónis meæ.}{望爾傾耳\\專聽予禱之音。}
\psVerse{Si iniquitátes observáveris, Dómine: * Dómine, quis sustinébit?}{主若憶人罪,\\主誰能堪之?}
\psVerse{Quia apud te propitiátio est: * et propter legem tuam sustínui te, Dómine.}{盖仁慈居於主。\\予為爾令忍受,倚厥諭。}
\psVerse{Sustínuit ánima mea in verbo ejus: * sperávit ánima mea in Dómino.}{予靈忍受,\\予靈專望主。}
\psVerse{A custódia matutína usque ad noctem: * speret Israël in Dómino.}{自晨更至夕,\\義撒厄爾已望主。}
\psVerse{Quia apud Dóminum misericórdia: * et copiósa apud eum redémptio.}{慈惻本在主,\\而救世之功遍滿矣。}
\psVerse{Et ipse rédimet Israël, * ex ómnibus iniquitátibus ejus.}{將救贖義撒厄爾\\於諸逆。}
\psGloria{Glória Patri, et Fílio, * et Spirítui Sancto. \\Sicut erat in princípio, et nunc, et semper, * et in sǽcula sæculórum. Amen.}{欽頌榮福於罷德肋、及費畧、及斯彼利多三多,\\若今茲、若永遠、及無窮世。亞孟。}
\psAntiphonRepeat{Apud Dóminum misericórdia: et copiósa apud eum redémptio.}{慈惻本在主,而救贖之功遍滿矣。}
\psAntiphonRepeat{De fructu ventris tui ponam super sedem tuam.}{由爾腹之實,將置爾座上。}
\psPsalmTitle{Psalmus 131}{聖詠131}
\psVerse{Meménto, Dómine, David, * et omnis mansuetúdinis ejus:}{主請記憶達未德,\\並厥諸良善。}
\psVerse{Sicut jurávit Dómino, * votum vovit Deo Jacob:}{如昔誓許主\\雅歌伯天主,}
\psVerse{Si introíero in tabernáculum domus meæ, * si ascéndero in lectum strati mei:}{予不敢突進內室,\\登榻設茵。}
\psVerse{Si dédero somnum óculis meis, * et pálpebris meis dormitatiónem:}{與閉予厥目,交睫假寐,\\枕藉憩息;}
\psVerse{Et réquiem tempóribus meis: donec invéniam locum Dómino, * tabernáculum Deo Jacob.}{至獲主安所,\\雅歌伯天主宮。}
\psVerse{Ecce audívimus eam in Éphrata: * invénimus eam in campis silvæ.}{我等聞在厄弗達,\\幸獲在林麓曠原。}
\psVerse{Introíbimus in tabernáculum ejus: * adorábimus in loco, ubi stetérunt pedes ejus.}{吾將進厥內室,\\敬拜履所。}
\psVerse{Surge, Dómine, in réquiem tuam, * tu et arca sanctificatiónis tuæ.}{主請舉趾安步,\\並携聖櫃。}
\psVerse{Sacerdótes tui induántur justítiam: * et sancti tui exsultatióne exsultábunt.}{望爾鐸德服義,\\衆聖歡躍。}
\psVerse{Propter David, servum tuum, * non avértas fáciem Christi tui.}{為爾僕達未德,\\勿背爾救世之面。}
\psVerse{Jurávit Dóminus David veritátem, et non frustrábitur eam: * De fructu ventris tui ponam super sedem tuam.}{昔定誓達未德,真實弗爽:\\由爾腹之實,將置爾座上。}
\psVerse{Si custodíerint fílii tui testaméntum meum, * et testimónia mea hæc, quæ docébo eos:}{若爾衆子遵予遺約,\\及將迪教誥厥子若孫,}
\psVerse{Et fílii eórum usque in sǽculum, * sedébunt super sedem tuam.}{世世\\坐爾座上。}
\psVerse{Quóniam elégit Dóminus Sion: * elégit eam in habitatiónem sibi.}{主已選西婉,\\定為厥居:}
\psVerse{Hæc réquies mea in sǽculum sǽculi: * hic habitábo quóniam elégi eam.}{予即安息於斯,世世將居,\\盖已選之。}
\psVerse{Víduam ejus benedícens benedícam: * páuperes ejus saturábo pánibus.}{將降福厥寡,\\以麫餅食厥貧。}
\psVerse{Sacerdótes ejus índuam salutári: * et sancti ejus exsultatióne exsultábunt.}{聖寵被撒責爾鐸德,\\暨衆聖忻愉踴躍。}
\psVerse{Illuc prodúcam cornu David, * parávi lucérnam Christo meo.}{彼達未德勇毅,\\於彼裂庭燎。}
\psVerse{Inimícos ejus índuam confusióne: * super ipsum autem efflorébit sanctificátio mea.}{敬基利斯督,戮辱厥仇,\\予寵綏於伊。}
\psGloria{Glória Patri, et Fílio, * et Spirítui Sancto. \\Sicut erat in princípio, et nunc, et semper, * et in sǽcula sæculórum. Amen.}{欽頌榮福於罷德肋、及費畧、及斯彼利多三多,\\若今茲、若永遠、及無窮世。亞孟。}
\psAntiphonRepeat{De fructu ventris tui ponam super sedem tuam.}{由爾腹之實,將置爾座上。}
\psRubric{Capitulum}{節目}
\psHeaderThree{Hebr. 1:1-2}{希一1,2}
\psText{Multifáriam, multisque modis olim Deus loquens pátribus in prophétis: novíssime diébus istis locútus est nobis in Fílio, quem constítuit heréde universórum, per quem fecit et sǽcula.}{昔天主屢次多方,啟先知者轉語吾祖;末時斯日,由厥子嗣萬有造寰宇者,語我等。}
\psVR{R}{Deo grátias.}{謝天主。}
\psRubric{Hymnus}{聖歌}
\psHymnHeader{Jesu, Redémptor ómnium}{救世耶穌}
\psHymnStanza{Jesu, Redémptor ómnium, Quem lucis ante oríginem, \\Parem patérnæ glóriæ, Pater suprémus édidit.}{救世耶穌,啟明之先,\\上父攸生,與已均榮。}
\psHymnStanza{Tu lumen, et splendor Patris, Tu spes perénnis ómnium: \\Inténde quas fundunt preces Tui per orbem sérvuli.}{爾乃父光,蒼生之望,\\普世微僕,攸祈俯聽。}
\psHymnStanza{Meménto, rerum Cónditor, Nostri quod olim córporis, \\Sacráta ab alvo Vírginis Nascéndo, formam súmpseris.}{肇基萬有,請記昔者,\\由童女胎,降取吾身。}
\psHymnStanza{Testátur hoc præsens dies, Currens per anni círculum, \\Quod solus a sinu Patris Mundi salus advéneris.}{週歲届期,今日作証,\\爾自父懷,降來救世。}
\psHymnStanza{Hunc astra, tellus, æquora, Hunc omne, quod cælo subest, \\Salútis auctórem novæ Novo salútant cántico.}{星地海幽,寰宇萬有,\\新咏讚主,復新人元。}
\psHymnStanza{Et nos, beáti, quos sacra Rigávit unda sánguinis, \\Natális ob diem tui, Hymni tribútum sólvimus.}{吾儕獲贖,因爾寶血,\\念爾聖誕,和賡稱頌。}
\psHymnStanza{Jesu, tibi sit glória, Qui natus es de Vírgine, \\Cum Patre et almo Spíritu, In sempitérna sǽcula. Amen.}{誕於童女,耶穌永福,\\歸爾偕父,聖神世世。亞孟。}
\psVR{V}{Notum fecit Dóminus, allelúja.}{主已顯亞勒路亞。}
\psVR{R}{Salutáre suum, allelúja.}{厥救世者亞勒路亞。}
\psRubric{Ad Magnificat}{聖母歌}
\psAntiphonRepeat{Hódie Christus natus est: hódie Salvátor appáruit: hódie in terra canunt Ángeli, lætántur Archángeli: hódie exsúltant justi, dicéntes: Glória in excélsis Deo, allelúja.}{今日基利斯督降誕,今日救世者顯。今日天神聲頌,宗使者忻樂。今日義人踴躍,曰:天主受享榮福於天。亞勒路亞。}
\psPsalmTitle{Magníficat (Luc. 1:46-55)}{聖母歌}
\psVerse{Magníficat ánima mea Dóminum: \\et exsultávit spíritus meus in Deo salutári meo.}{感頌吾主吾神,\\無任忻愉於救我者。}
\psVerse{Quia respéxit humilitátem ancíllæ suæ: * \\ecce enim ex hoc beátam me dicent omnes generatiónes.}{緣其垂顧婢子之微,\\後人亦將於我乎讚頌矣。}
\psVerse{Quia fecit mihi magna, qui potens est: * \\et sanctum nomen ejus.}{夫全能者大展厥德於我,\\錫以異恩用彰聖名。}
\psVerse{Et misericórdia ejus, a progénie in progénies: * \\timéntibus eum.}{仁慈無量,將沿世世,\\於諸畏敬之者。}
\psVerse{Fecit poténtiam in bráchio suo: * \\dispérsit supérbos mente cordis sui.}{以厥臂神力顯大能,\\麾彼驕盈。}
\psVerse{Depósuit poténtes de sede: * \\et exaltávit húmiles.}{黜彼尊者於高位,\\而陟舉夫謙遜者。}
\psVerse{Esuriéntes implévit bonis: * \\et dívites dimísit inánes.}{饑虛以福實之,\\飫滿以傾棄之。}
\psVerse{Suscépit Israël púerum suum: * \\recordátus misericórdiæ suæ.}{且不忘大慈,\\賜以其子,}
\psVerse{Sicut locútus est ad patres nostros: * \\Ábraham, et sémini ejus in sǽcula.}{以踐所許於吾祖\\亞罷郎,及後世之子孫者。}
\psGloria{Glória Patri, et Fílio, * et Spirítui Sancto. \\Sicut erat in princípio, et nunc, et semper, * et in sǽcula sæculórum. Amen.}{欽頌榮福於罷德肋、及費畧、及斯彼利多三多,\\若今茲、若永遠、及無窮世。亞孟。}
\psAntiphonRepeat{Hódie Christus natus est: hódie Salvátor appáruit: hódie in terra canunt Ángeli, lætántur Archángeli: hódie exsúltant justi, dicéntes: Glória in excélsis Deo, allelúja.}{今日基利斯督降誕,今日救世者顯。今日天神聲頌,宗使者忻樂。今日義人踴躍,曰:天主受享榮福於天。亞勒路亞。}
\psRubric{Oratio}{祝文}
\psVR{V}{Dóminus vobíscum.}{主與爾偕焉。}
\psVR{R}{Et cum spíritu tuo.}{並於爾神。}
\psCollect{Concéde, quǽsumus, omnípotens Deus: ut nos Unigéniti tui nova per carnem Natívitas líberet; quos sub peccáti jugo vetústa sérvitus tenet. Per eúndem Dóminum nostrum Jesum Christum Filium tuum: Qui tecum vivit et regnat in unitate Spiritus Sancti Deus, per omnia saecula saeculorum.}{懇祈全能天主,彼久在罪軛,賴爾惟一子肉身新誕,幸救脫亦。為聖子耶穌基利斯督我等主,其偕爾偕斯彼利多三多,為一天主,永生永王。}
\psVR{R}{Amen.}{亞孟。}
\psHeaderThree{Commemoratio S. Stephani Protomartyris}{聖斯德望首致命憶文}
\psAntiphonRepeat{Stéphanus autem plenus grátia et fortitúdine, faciébát prodígia et signa magna in pópulo.}{斯德望充滿聖寵,以勇敢行多靈異,廣奇於衆。}
\psVR{V}{Glória et honóre coronásti eum, Dómine.}{主以崇福冕旒已加伊首。}
\psVR{R}{Et constituísti eum super ópera mánuum tuárum.}{且置伊爾手諸工上。}
\psCollect{Da nobis, quǽsumus, Dómine, imitári quod cólimus, ut discámus et inimícos dilígere: quia ejus natalítia celebrámus, qui novit étiam pro persecutóribus exoráre Dóminum nostrum Jesum Christum Fílium tuum: Qui tecum vivit et regnat in unitáte Spíritus Sancti Deus, per ómnia sǽcula sæculórum.}{祈望天主俾吾攸敬即師法也,克愛吾仇。斯誕日敬彼為讐求主,為爾子耶穌基利斯督我等主,其偕爾偕斯彼利多三多,為一天主,永生永王。}
\psVR{R}{Amen.}{亞孟。}
\psVR{V}{Dóminus vobíscum.}{主與爾偕焉。}
\psVR{R}{Et cum spíritu tuo.}{並於爾神。}
\psVR{V}{Benedicámus Dómino.}{讚美主。}
\psVR{R}{Deo grátias.}{謝天主。}
\psVR{V}{Fidélium ánimæ per misericórdiam Dei requiéscant in pace.}{凡諸信者靈魂,賴天主仁慈息止安所。}
\psVR{R}{Amen.}{亞孟。}
\psRubric{(Pater noster, secreto.)}{(後默念天主經)}
\psEnterSingleCol
\psSinglePageBreak
\psSingleHeaderTwo{附注}
\psSingleText{1.本聖誕日課爲羅馬大日課項目試驗品,目前沒有取得“Imprimatur/主教准”,不具有官方性質。封面頁已註明“Pro Manuscripto”。\\2.其中拉丁文文本取自https://www.divinumofficium.com/,見https://github.com/DivinumOfficium/divinum-officium;中文文本取自利類思(Lodovico Buglio)譯《日課概要》(又名《司鐸課典》,見https://bureaudumercure.org/mercure/node/838699;\\3.校勘:\\3.1 夜課經第三節經書八、九前降福文本經核實有誤,依拉丁文文本重譯。原作:Per evangélica dicta deleántur nostra delícta.幸因萬日畧經滌除吾孽;Ad societátem cívium supernórum perdúcat nos Rex Angelórum.望基利斯督天主子誨我等萬日畧云;\\3.2 讚美經聖歌 A solis ortus cárdine 自日出樞至地末境中; Gaudet chorus cæléstium Et Ángeli canunt Deo; Palamque fit pastóribus Pastor Creátor ómnium.天神預示,若望覺知;母胎踴躍,童女誕育。此節中文與拉丁文不完全對應,依中文本保留;Eníxa est puérpera Quem Gábriel prædíxerat: Quem Matris alvo géstiens Clausus Joánnes sénserat.置身草茅,不棄馬櫪。養育蒼生,以微乳哺。此節拉丁文意為:若望在母胎已知...;\\3.3 一時經 (晨經)中,啓:Qui natus es de María Vírgine.生於童女瑪利亞。中文文本爲“坐於聖父之右者”,據禮規改"\\4.簡短術語對照(待擴充):\\Patri-罷德肋-聖父; Fílio-費畧-聖子; Spirítui Sancto-斯彼利多三多-聖神; Christus-基利斯督-基督; Allelúja-亞勒路亞-阿肋路亞; Vesperas-申正經-晚禱; Matutínum-夜課經-誦讀日課; Laudes-讚美經-晨禱; Primam-一時經-首時; Tertiam-三時經-午前禱; Sextam-六時經-午間禱; Nonam-九時經-午後禱; Glória-欽頌榮福-光榮頌; Mártyrum-致命-殉道者; Sion-西婉-熙雍; Melchísedech-默其塞德-默基瑟德; Sacérdos-撒責爾鐸-司鐸; \\Et cum spíritu tuo-並於爾神-也與你的心靈同在; Dóminus vobíscum-主與爾偕焉-願主與你們同在; \\Deus in adiutórium meum inténde-天主惟專於我扶祐-天主,求你快來拯救我; \\Dómine ad adjuvándum me festína-主速格以救助我-上主,求你速來扶助我; \\Fidélium ánimæ per misericórdiam Dei requiéscant in pace-凡諸信者靈魂,賴天主仁慈息止安所-願信友的靈魂,賴天主的仁慈,息止安所。\\這份文件若有寫的不專業的地方,還望讀者諒解和指正。"}
\psSingleHeaderTwo{Scripsit Clemens B. die XXII Decémbris MMXXV.}
\psExitSingleCol